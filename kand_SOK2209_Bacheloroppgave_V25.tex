% Options for packages loaded elsewhere
\PassOptionsToPackage{unicode}{hyperref}
\PassOptionsToPackage{hyphens}{url}
\PassOptionsToPackage{dvipsnames,svgnames,x11names}{xcolor}
%
\documentclass[
  12pt,
  a4paper,
  DIV=11,
  numbers=noendperiod]{scrartcl}

\usepackage{amsmath,amssymb}
\usepackage{iftex}
\ifPDFTeX
  \usepackage[T1]{fontenc}
  \usepackage[utf8]{inputenc}
  \usepackage{textcomp} % provide euro and other symbols
\else % if luatex or xetex
  \usepackage{unicode-math}
  \defaultfontfeatures{Scale=MatchLowercase}
  \defaultfontfeatures[\rmfamily]{Ligatures=TeX,Scale=1}
\fi
\usepackage{lmodern}
\ifPDFTeX\else  
    % xetex/luatex font selection
\fi
% Use upquote if available, for straight quotes in verbatim environments
\IfFileExists{upquote.sty}{\usepackage{upquote}}{}
\IfFileExists{microtype.sty}{% use microtype if available
  \usepackage[]{microtype}
  \UseMicrotypeSet[protrusion]{basicmath} % disable protrusion for tt fonts
}{}
\makeatletter
\@ifundefined{KOMAClassName}{% if non-KOMA class
  \IfFileExists{parskip.sty}{%
    \usepackage{parskip}
  }{% else
    \setlength{\parindent}{0pt}
    \setlength{\parskip}{6pt plus 2pt minus 1pt}}
}{% if KOMA class
  \KOMAoptions{parskip=half}}
\makeatother
\usepackage{xcolor}
\usepackage[top=20mm,left=20mm,heightrounded]{geometry}
\setlength{\emergencystretch}{3em} % prevent overfull lines
\setcounter{secnumdepth}{-\maxdimen} % remove section numbering
% Make \paragraph and \subparagraph free-standing
\ifx\paragraph\undefined\else
  \let\oldparagraph\paragraph
  \renewcommand{\paragraph}[1]{\oldparagraph{#1}\mbox{}}
\fi
\ifx\subparagraph\undefined\else
  \let\oldsubparagraph\subparagraph
  \renewcommand{\subparagraph}[1]{\oldsubparagraph{#1}\mbox{}}
\fi


\providecommand{\tightlist}{%
  \setlength{\itemsep}{0pt}\setlength{\parskip}{0pt}}\usepackage{longtable,booktabs,array}
\usepackage{calc} % for calculating minipage widths
% Correct order of tables after \paragraph or \subparagraph
\usepackage{etoolbox}
\makeatletter
\patchcmd\longtable{\par}{\if@noskipsec\mbox{}\fi\par}{}{}
\makeatother
% Allow footnotes in longtable head/foot
\IfFileExists{footnotehyper.sty}{\usepackage{footnotehyper}}{\usepackage{footnote}}
\makesavenoteenv{longtable}
\usepackage{graphicx}
\makeatletter
\def\maxwidth{\ifdim\Gin@nat@width>\linewidth\linewidth\else\Gin@nat@width\fi}
\def\maxheight{\ifdim\Gin@nat@height>\textheight\textheight\else\Gin@nat@height\fi}
\makeatother
% Scale images if necessary, so that they will not overflow the page
% margins by default, and it is still possible to overwrite the defaults
% using explicit options in \includegraphics[width, height, ...]{}
\setkeys{Gin}{width=\maxwidth,height=\maxheight,keepaspectratio}
% Set default figure placement to htbp
\makeatletter
\def\fps@figure{htbp}
\makeatother
% definitions for citeproc citations
\NewDocumentCommand\citeproctext{}{}
\NewDocumentCommand\citeproc{mm}{%
  \begingroup\def\citeproctext{#2}\cite{#1}\endgroup}
\makeatletter
 % allow citations to break across lines
 \let\@cite@ofmt\@firstofone
 % avoid brackets around text for \cite:
 \def\@biblabel#1{}
 \def\@cite#1#2{{#1\if@tempswa , #2\fi}}
\makeatother
\newlength{\cslhangindent}
\setlength{\cslhangindent}{1.5em}
\newlength{\csllabelwidth}
\setlength{\csllabelwidth}{3em}
\newenvironment{CSLReferences}[2] % #1 hanging-indent, #2 entry-spacing
 {\begin{list}{}{%
  \setlength{\itemindent}{0pt}
  \setlength{\leftmargin}{0pt}
  \setlength{\parsep}{0pt}
  % turn on hanging indent if param 1 is 1
  \ifodd #1
   \setlength{\leftmargin}{\cslhangindent}
   \setlength{\itemindent}{-1\cslhangindent}
  \fi
  % set entry spacing
  \setlength{\itemsep}{#2\baselineskip}}}
 {\end{list}}
\usepackage{calc}
\newcommand{\CSLBlock}[1]{\hfill\break\parbox[t]{\linewidth}{\strut\ignorespaces#1\strut}}
\newcommand{\CSLLeftMargin}[1]{\parbox[t]{\csllabelwidth}{\strut#1\strut}}
\newcommand{\CSLRightInline}[1]{\parbox[t]{\linewidth - \csllabelwidth}{\strut#1\strut}}
\newcommand{\CSLIndent}[1]{\hspace{\cslhangindent}#1}

\KOMAoption{captions}{tableheading}
\usepackage{wrapfig}
\usepackage{subcaption}
\usepackage{amsmath}
\usepackage{cancel}
\usepackage{hyperref}
\usepackage{tikz}
\usetikzlibrary{shapes.geometric, arrows, arrows.meta, positioning, calc}
\usepackage{tabularx}
\renewcommand{\maketitle}{}
\usepackage{fancyhdr}
\pagestyle{fancy}
\fancyhf{}
\fancyhead[L]{\rightmark}
\fancyhead[R]{\thepage}
\fancyfoot[C]{\thepage}
\usepackage{colortbl}
\definecolor{cornflowerblue}{RGB}{100,149,237}
\definecolor{darkblue}{RGB}{115,150,255}
\definecolor{lighterblue}{RGB}{131, 191, 212}
\definecolor{lightblue}{RGB}{178,211,220}
\makeatletter
\@ifpackageloaded{caption}{}{\usepackage{caption}}
\AtBeginDocument{%
\ifdefined\contentsname
  \renewcommand*\contentsname{Table of contents}
\else
  \newcommand\contentsname{Table of contents}
\fi
\ifdefined\listfigurename
  \renewcommand*\listfigurename{Figurliste}
\else
  \newcommand\listfigurename{Figurliste}
\fi
\ifdefined\listtablename
  \renewcommand*\listtablename{Tabelliste}
\else
  \newcommand\listtablename{Tabelliste}
\fi
\ifdefined\figurename
  \renewcommand*\figurename{Figur}
\else
  \newcommand\figurename{Figur}
\fi
\ifdefined\tablename
  \renewcommand*\tablename{Tabell}
\else
  \newcommand\tablename{Tabell}
\fi
}
\@ifpackageloaded{float}{}{\usepackage{float}}
\floatstyle{ruled}
\@ifundefined{c@chapter}{\newfloat{codelisting}{h}{lop}}{\newfloat{codelisting}{h}{lop}[chapter]}
\floatname{codelisting}{Listing}
\newcommand*\listoflistings{\listof{codelisting}{List of Listings}}
\makeatother
\makeatletter
\makeatother
\makeatletter
\@ifpackageloaded{caption}{}{\usepackage{caption}}
\@ifpackageloaded{subcaption}{}{\usepackage{subcaption}}
\makeatother
\ifLuaTeX
  \usepackage{selnolig}  % disable illegal ligatures
\fi
\usepackage{bookmark}

\IfFileExists{xurl.sty}{\usepackage{xurl}}{} % add URL line breaks if available
\urlstyle{same} % disable monospaced font for URLs
\hypersetup{
  colorlinks=true,
  linkcolor={blue},
  filecolor={Maroon},
  citecolor={Blue},
  urlcolor={Blue},
  pdfcreator={LaTeX via pandoc}}

\author{}
\date{}

\begin{document}


\newgeometry{left=0cm, right=0cm, top=0cm, bottom=0cm}
\vspace*{0.5cm} 
\hspace*{1.5cm}\includegraphics[width=10cm]{dokumentobjekter/texstuff/UiT_Logo_Bok_Bla_RGB.png} 


\begin{flushleft}
    \vspace*{0.5cm}
    \hspace*{2.5cm}{\color{black}\fontsize{11}{13.2}\selectfont Handelshøgskolen ved UiT \\[0.2em]
    \hspace*{2.5cm}\color{black}\fontsize{8}{13.2}\selectfont Fakultet for biovitenskap, fiskeri og økonomi \\[0.2em]
    \hspace*{2.5cm}\large{\color{black}\textbf{navn på oppgave}}  \\[0.5em]
    \hspace*{2.5cm}\color{black}\fontsize{12}{14.4}\selectfont Bacheloroppgave i: SOK-2209 Bacheloroppgave i samfunnsøkonomi - 20 stp  \\[0.5em]
\hspace*{2.5cm}\color{black}\fontsize{11}{13.2}\selectfont Kandidatnummer: Daniel og Daniel \\[0.5em]
    \hspace*{2.5cm}\color{black}\fontsize{11}{13.2}\selectfont Sok-2209, Vår 2025 \\[0.5em]
    \hspace*{2.0cm}
    \par}
\end{flushleft} 



\begin{tikzpicture}[remember picture, overlay]
    \node[anchor=south west, inner sep=0] at (current page.south west) {\includegraphics[width=\paperwidth]{dokumentobjekter/texstuff/forside_bilde.png}};
\end{tikzpicture}


\newgeometry{left=20mm, right=20mm, top=20mm, bottom=20mm}





\thispagestyle{plain}
\begin{center}
    \Large
    \textbf{Forord}
\end{center}






\newpage
\hypersetup{linkcolor=black}
\renewcommand{\contentsname}{Innholdsfortegnelse}
\renewcommand*{\figureautorefname}{Figur}
\renewcommand*{\tableautorefname}{Tabell}
\tableofcontents
\listoffigures
\listoftables
\hypersetup{linkcolor=blue}
\newpage

\section{1 Innledning}\label{innledning}

Redegjør for temaområde og oppgavens plass i dette. Presentere
problemstilling og hvorfor denne er interessant å finne ut av.

\subsection{a) Hva omhandler oppgaven -- Beskriv temaet i én eller to
setninger. Dette kan være hele
problemstillingen.}\label{a-hva-omhandler-oppgaven-beskriv-temaet-i-uxe9n-eller-to-setninger.-dette-kan-vuxe6re-hele-problemstillingen.}

Forklarer nivået på formue sykefraværet i Norge?

Denne bacheloroppgaven undersøker sammenhengen mellom sosioøkonomiske
forhold og sykefravær. Vi ser på effekten formuenivået påvirker
arbeidstakeres helse og fravær fra jobben.

Vi benytter en Job Demands-Resources (JDR) modell som teoretisk
rammeverk, og analyserer data fra Levekårsundersøkelsen om arbeidsmiljø.

\subsection{b) Motiver hvorfor temaet er viktig å studere fra et
samfunns(økonomisk)
perspektiv}\label{b-motiver-hvorfor-temaet-er-viktig-uxe5-studere-fra-et-samfunnsuxf8konomisk-perspektiv}

De siste årene i Norge har sykefraværet hatt en økende trend. Det har
vært debatt i media om hvordan man kan redusere sykefraværet.

I årene etter finanskrisen har vi observert en økende formueulikhet i
mange vestlige land, inkludert Norge. Denne trenden har blitt forsterket
etter pandemien, spesielt i boligmarkedet. Lønnsveksten har ikke holdt
tritt med prisøkningen på eiendeler, særlig boliger. Dette har gjort det
relativt vanskeligere for unge og de med lavere inntekter å opparbeide
seg formue, for eksempel gjennom boligkjøp.

Vi antar dermed at formuenivået til arbeidstakere har en effekt på
spesielt motivasjon og helse, og dermed påvirker sykefraværet. Når det
blir stadig vanskeligere å oppnå økonomisk trygghet og en akseptabel
levestandard, kan det føre til økt stress, redusert jobbmotivasjon, og i
verste fall dårligere helse og økt fravær.

Å forstå disse mekanismene er viktig for å utvikle treffsikre tiltak for
å redusere sykefravær og fremme et inkluderende arbeidsliv.

\subsection{c) Evt. Gi en kort bakgrunn til temaet i
Norge}\label{c-evt.-gi-en-kort-bakgrunn-til-temaet-i-norge}

Kommentar: Rekkefølgen på b og c kan endres.

\section{2. Teoretisk grunnlag}\label{teoretisk-grunnlag}

\subsection{2.1 Definere begrepene
kortfattet}\label{definere-begrepene-kortfattet}

\subsubsection{a) Sykefravær}\label{a-sykefravuxe6r}

\subsubsection{b) Formue}\label{b-formue}

\subsubsection{c) Job Demands-Resources (JDR)
modell}\label{c-job-demands-resources-jdr-modell}

\paragraph{Jobbkrav}\label{jobbkrav}

\paragraph{Jobbressurser}\label{jobbressurser}

\subsection{2.2 Modelloppsett}\label{modelloppsett}

\subsection{Gjør rede for relevant teori på temaområdet med særlig fokus
på den teorien dere har valgt å benytte for å svare på
problemstillingen.}\label{gjuxf8r-rede-for-relevant-teori-puxe5-temaomruxe5det-med-suxe6rlig-fokus-puxe5-den-teorien-dere-har-valgt-uxe5-benytte-for-uxe5-svare-puxe5-problemstillingen.}

Forklaring av hvorfor dette er viktig for samfunnet

\section{3. Tidligere forskning}\label{tidligere-forskning}

\subsubsection{3.1 Forklar kortfattet hva tidligere forskning har funnet
generelt om problemstillingen (hvorfor det er viktig å studere fra et
sammfunsperspektiv)}\label{forklar-kortfattet-hva-tidligere-forskning-har-funnet-generelt-om-problemstillingen-hvorfor-det-er-viktig-uxe5-studere-fra-et-sammfunsperspektiv}

\subsubsection{3.2 Forklar teori og empiriske funn knyttet til koblingen
som du vil undersøke. Være nøye med å gjøre rede for
mekanismer!}\label{forklar-teori-og-empiriske-funn-knyttet-til-koblingen-som-du-vil-undersuxf8ke.-vuxe6re-nuxf8ye-med-uxe5-gjuxf8re-rede-for-mekanismer}

\section{4. Hypoteser}\label{hypoteser}

\section{5. Metode og datagrunnlag}\label{metode-og-datagrunnlag}

Forklar og begrunn metodevalg i henhold til problemstillingen. Beskriv
hvordan data til oppgaven er fremskaffet. Gi en innledende oversikt over
data (deskriptiv statistikk).

\subsection{5.1 Datamaterial}\label{datamaterial}

\subsection{5.1..1 Datakilde og utvalg}\label{datakilde-og-utvalg}

\subsection{5.1..2 Variabler}\label{variabler}

\subsubsection{a) Avhengig variabel}\label{a-avhengig-variabel}

\subsubsection{b) Variabler knyttet til
hypotese}\label{b-variabler-knyttet-til-hypotese}

\subsubsection{c) Kontrollvariabler}\label{c-kontrollvariabler}

Kommentar: Deskriptiv statistikk kan evt legges til som avsnitt 5.1.3

\subsection{5.2 Økonometrisk modell}\label{uxf8konometrisk-modell}

\subsubsection{a) Ligning til modellen}\label{a-ligning-til-modellen}

\subsubsection{b) Forklaring av alle deler i
modellen}\label{b-forklaring-av-alle-deler-i-modellen}

\subsubsection{c) Beskrivning av økonometrisk
metode}\label{c-beskrivning-av-uxf8konometrisk-metode}

Kommentar: C kan tas først eller sist.

\section{6. Resultat}\label{resultat}

Her presenteres den empiriske analysen og dens resultater. Vanligvis vil
en empirisk analyse bestå av en regresjonsanalyse med flere variabler.
Andre muligheter kan diskuteres med veilederen.

\subsection{6.1 Deskriptiv statistikk}\label{deskriptiv-statistikk}

\subsubsection{a) Tabell med alle variabler som er tatt med i
analysen}\label{a-tabell-med-alle-variabler-som-er-tatt-med-i-analysen}

\subsubsection{b) Eventuelt figurer over «problematiske»
variabler}\label{b-eventuelt-figurer-over-problematiske-variabler}

\subsubsection{c) Kommentarer til
datamaterialet}\label{c-kommentarer-til-datamaterialet}

Kommentar: avsnitt 6.1 kan flyttes til et avsnitt 5.1.3

\subsection{6.2 Økonometrisk analyse}\label{uxf8konometrisk-analyse}

\subsubsection{a) Tabell med resultat fra
regresjonsanalysen(e)}\label{a-tabell-med-resultat-fra-regresjonsanalysene}

\subsubsection{b) Redegjørelse for resultat knyttet til
hypoteser}\label{b-redegjuxf8relse-for-resultat-knyttet-til-hypoteser}

\subsubsection{c) Redegjørelse for effekt av
kontrollvariabler}\label{c-redegjuxf8relse-for-effekt-av-kontrollvariabler}

\subsubsection{d) Redegjørelse for svakheter i
modellen/data}\label{d-redegjuxf8relse-for-svakheter-i-modellendata}

\section{7. Diskusjon}\label{diskusjon}

Dette kapitlet drøfter resultatene i forhold til problemstillingen. Hva
er funnet ut av, hva gjenstår, hvilke styrker og svakheter har analysen?

\subsection{a) Oppsummering av hva formålet med oppgaven var, og hva
analysen
viste}\label{a-oppsummering-av-hva-formuxe5let-med-oppgaven-var-og-hva-analysen-viste}

\subsection{b) Diskusjon av hvilke konklusjoner som kan trekkes fra
dette og om resultatene er forenlig med tidligere
funn/teori}\label{b-diskusjon-av-hvilke-konklusjoner-som-kan-trekkes-fra-dette-og-om-resultatene-er-forenlig-med-tidligere-funnteori}

\subsection{c) Diskusjon av svakheter i
analysen}\label{c-diskusjon-av-svakheter-i-analysen}

\subsection{d) Diskusjon av implikasjoner for policy gitt
svakheter}\label{d-diskusjon-av-implikasjoner-for-policy-gitt-svakheter}

\subsection{e) Eventuelt: diskusjon av hva framtidig forskning kan
forske videre på (basert påderes funn og svakheter i
analysen)}\label{e-eventuelt-diskusjon-av-hva-framtidig-forskning-kan-forske-videre-puxe5-basert-puxe5deres-funn-og-svakheter-i-analysen}

\clearpage

\section{Referanser}\label{referanser}

\phantomsection\label{refs}
\begin{CSLReferences}{1}{0}
\bibitem[\citeproctext]{ref-jaeggi2021wealth}
Jaeggi, A. V., Blackwell, A. D., Von Rueden, C., Trumble, B. C.,
Stieglitz, J., Garcia, A. R., Kraft, T. S., Beheim, B. A., Hooper, P.
L., Kaplan, H., et al. (2021). Do wealth and inequality associate with
health in a small-scale subsistence society? \emph{Elife}, \emph{10},
e59437.

\end{CSLReferences}

\clearpage

\appendix

\section {Appendix Generell KI bruk}

I løpet av koden så kan det ses mange \# kommentarer der det er skrevet
for eks ``\#fillbetween q1 and q2''. Når vi skriver kode i Visual Studio
Code og Rstudio så er det en plugin som heter Github Copilot. Når vi
skriver slike kommentarer eller bare skriver kode så kan den foresøke å
fullføre kodelinjene mens vi skriver de. Noen ganger klarer den det, men
andre ikke. Det er vanskelig å dokumentere hvert bruk der den er brukt
siden det ``går veldig fort'' men siden vi ikke har fått på plass en
slik dokumentasjon så kan all kode der det er brukt kommentarer antas
som at det er brukt Github Copilot. Nærmere info om dette KI verktøyet
kan ses på \url{https://github.com/features/copilot}

\clearpage

\subsection{13760 Arbeidsstyrken, sysselsatte, arbeidsledige og utførte
ukeverk for personer 15-74 år, etter kjønn, alder, type justering, måned
og
statistikkvariabel}\label{arbeidsstyrken-sysselsatte-arbeidsledige-og-utfuxf8rte-ukeverk-for-personer-15-74-uxe5r-etter-kjuxf8nn-alder-type-justering-muxe5ned-og-statistikkvariabel}

\subsection{12439: Sykefravær for lønnstakere(prosent), etter kjønn,
type sykefravær, kvartal og
statistikkvariabel}\label{sykefravuxe6r-for-luxf8nnstakereprosent-etter-kjuxf8nn-type-sykefravuxe6r-kvartal-og-statistikkvariabel}

\section{notater}\label{notater}

er det avvik mellom fastsatt arbeidstid og hvor mye folk arbeider?

Er folk overarbeidet?

Hvordan har antall legebesøk endret seg samtidig som legemeldt
sykefravær har endret seg. er leger overarbeidet og skriver ut for mye
sykefravær?

Er sykefraværet et problem? Hvordan har sysselsettingsrate endret seg
med sykefravær? er det spesiell korrelasjon mellom egenmeldt og
legemeldt der?

Dårlig ledelse og lite engasjerte arbeidere?

https://www.dagensperspektiv.no/synspunkt/benedicte-langseth-eide-svarer-hr-norge-om-sykefravaer-og-ledelse/1262876

https://www.nord24.no/nar-bedriftene-sliter-med-hoyt-sykefravar-ringer-de-benedicte-disse-tiltakene-nytter/s/5-32-197683

https://www.mdpi.com/1660-4601/18/8/4327

The results provide longitudinal evidence that two well-established job
resources (i.e., social support and feedback) predicted work engagement,
that work engagement was negatively related to sick leave and that this
relation was mediated by subjective health. By showing that
health-related indicators could also be outcomes of the motivational
process in the JD-R model, we have strengthened the model.

https://munin.uit.no/handle/10037/15801

The results also revealed that both workaholics and work-engaged
employees put in more hours at work than was expected of them. We found
that workaholism was negatively related to work-related health, whereas
work engagement was positively related to work-related health. These
findings support the notion of workaholism and work engagement as two
different forms of working hard.

Kanskje en form for ``intensitet'' i hvor sensitiv du er.

Jeg tror formue spiller inn til hvor sensitiv du er til endringer i
inntekt. Altså ditt konsumnnivå eller etterspurt fritid endrer seg ulikt
basert på om du har mye formue eller ikke. Dette kan være fordi du har
mer buffer til å tåle endringer i inntekt.

trur vi blir å få noe bue på den effekten. fattige, vanlige, rike,
megarike vil ha ulik effekt av motivasjon og sånt. e du megarik så har
det jo ingenting og si, e du syk eller vil ta fri så blir du hjemme, men
samtidig så vil du kanskje være spesielt sensitiv om du e fattig og at
om du da e syk eller vil ta fri så vil du både ha dårligere utgangspunkt
i jobbtype og sånt, og også kunne rett å slett være mer syk

mens de i midten rundt ``vanlige'' mot bare rike kan ha 0 effekt, men
kommer vel an på kor mye man ska mene formue har å si til hvor sensitiv
du er til endringer eller potentielle endringer i inntekt derfor æ
tenkte å bare ha det til å være en funksjon av formue kunne være enklere
motivasjon og sånt altså både på bunn og på topp så vil du også ha økt
den stygge m'en ved at du får statlige overføringe som fattig men mye
kapitalfortjeneste som rik

så formue har effekt på hvor mye utdanning du har. formue har effekt på
hvilken motivasjon du har. formue har effekt på m som er annen inntekt
utenom jobb.

g = formue, j = alder, k = utdanning, l = motivasjon

\begin{itemize}
\tightlist
\item
  v = dummyvariabel
\end{itemize}

\[
t^a = h^* - \alpha w - \beta(m(g) + h^*w) - (k\cdot v+j\cdot v)
\]

Dummy variabler for ulike aldersgrupper. beholde en ligning for alle men
da bruke de dummyvariablene. dermed kunne tolke bare en variabel.

forskjellige typer inntekt påvirke forskjellig i m variabelen.

Grunn til cb er at den er enkel og at vi nesten alltid tar log av
dataen. om vi har 0 variabler så blir det bare tull.

\subsection{Notater}\label{notater-1}

\subsubsection{Inntektsfattigdom og
levekårsfattigdom}\label{inntektsfattigdom-og-levekuxe5rsfattigdom}

https://www.ssb.no/sosiale-forhold-og-kriminalitet/artikler-og-publikasjoner/inntektsfattig-eller-levekaarsfattig

Hva så med en mer absolutt tilnærming i form av et forbruksbudsjett
inkludert faktiske bokostnader? Det enkleste målet som ikke tar hensyn
til verken studenter eller formuende, har omtrent like sterk sammenheng
med levekårsfattigdom som den vi finner ved EU60, og dermed noe sterkere
enn ved OECD50. Ved å holde studenter og/eller formuende utenfor
definisjonen med budsjettilnærming, får vi de samme virkningene som vi
har sett tidligere. Det å holde formuende utenfor bidrar til sterkere
sammenheng med levekårsfattigdom, mens det å holde studenter utenfor
ikke gjør det.

Våre funn viser dermed at det ikke er avgjørende om vi definerer
inntektsfattigdom absolutt (ved bruk av husholdningsbudsjett) eller
relativt (ved bruk av ekvivalensskala og inntektsfordeling) når vi ser
på sammenhengen med levekårsfattigdom. Den viktigste faktoren synes å
være at vi tar hensyn til formue, som er en buffer mot mange av
levekårsproblemene. Det har imidlertid ikke særlig betydning å ta hensyn
til studenter i denne sammenhengen, selv om det bidrar til å redusere
andelen inntektsfattige

\subsubsection{Helse og formue}\label{helse-og-formue}

https://pmc.ncbi.nlm.nih.gov/articles/PMC8225390/

controlling for community-average wealth, age, sex, household size,
community size, and distance to markets. Wealthier people largely had
better outcomes while inequality associated with more respiratory
disease, a leading cause of mortality. Greater inequality and lower
wealth were associated with higher blood pressure. Psychosocial factors
did not mediate wealth-health associations. Thus, relative
socio-economic position and inequality may affect health across diverse
societies, though this is likely exacerbated in high-income countries.

\subsubsection{Gatsby curve}\label{gatsby-curve}

``great gatsby curve'' med vedvarende inntekt på tvers av generasjoner.
og siden fattige ikke blir spesielt mye fattigere enn middelklassen, men
at det heller er rikere som flyr fremover. -\textgreater{} kanskje
større forskjell på median og gjennomsnittlig inntekt/formue altså flere
som er ikke rike som jobber i mer sånn lav inntekt yrker og barnehager å
sånt me mye sykdom

\subsubsection{karriærevalg, utdanning
osv.}\label{karriuxe6revalg-utdanning-osv.}

fattigere har dårligere tilgang på ``career role models'' som gjør at de
kanskje ikke vet om de bedre yrkene og sånt og dermed igjen blir mindre
utdanna og sånt
https://www.gallup.com/analytics/506696/amazon-research-hub.aspx

\subsubsection{Stress? glemte studie
her}\label{stress-glemte-studie-her}

inntektsusikkerhet -\textgreater{} økt stress

\subsubsection{Motivasjonseffekt av
ulikhet}\label{motivasjonseffekt-av-ulikhet}

``The motivational cost of inequality: Opportunity gaps reduce the
willingness to work'' https://pmc.ncbi.nlm.nih.gov/articles/PMC7473543/

https://www.brookings.edu/articles/income-inequality-social-mobility-and-the-decision-to-drop-out-of-high-school/

ulikhet gjør at fattige blir mindre motivert siden dem føler det å bli
rik er ``umulig'' og dermed investerer mindre i seg -\textgreater{}
lavere motivasjon og lavere utdanning. kanskje mer fysisk arbeid.

\subsection{Notater}\label{notater-2}

Har høy/lav formue effekt på motivasjonen fra lønnen til arbeid. lav
formue + høy lønn = høy motivasjon? høy formue + høy lønn = ``bryr meg
ikke'' = høy formue+ lav lønn, lav formue + lav lønn = lav motivajon

Kapitalinntekter som rente/aksje osv i forhold til bruttofinanskapital i
alt. kan det være at de med høy formue utenom bolig da har mer andre
inntekter, eller at høy formue bare er lik bolig for mange.



\end{document}
