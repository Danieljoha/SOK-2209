% Options for packages loaded elsewhere
\PassOptionsToPackage{unicode}{hyperref}
\PassOptionsToPackage{hyphens}{url}
\PassOptionsToPackage{dvipsnames,svgnames,x11names}{xcolor}
%
\documentclass[
  12pt,
  a4paper,
  DIV=11,
  numbers=noendperiod]{scrartcl}

\usepackage{amsmath,amssymb}
\usepackage{iftex}
\ifPDFTeX
  \usepackage[T1]{fontenc}
  \usepackage[utf8]{inputenc}
  \usepackage{textcomp} % provide euro and other symbols
\else % if luatex or xetex
  \usepackage{unicode-math}
  \defaultfontfeatures{Scale=MatchLowercase}
  \defaultfontfeatures[\rmfamily]{Ligatures=TeX,Scale=1}
\fi
\usepackage{lmodern}
\ifPDFTeX\else  
    % xetex/luatex font selection
  \setmainfont[]{Times New Roman}
\fi
% Use upquote if available, for straight quotes in verbatim environments
\IfFileExists{upquote.sty}{\usepackage{upquote}}{}
\IfFileExists{microtype.sty}{% use microtype if available
  \usepackage[]{microtype}
  \UseMicrotypeSet[protrusion]{basicmath} % disable protrusion for tt fonts
}{}
\makeatletter
\@ifundefined{KOMAClassName}{% if non-KOMA class
  \IfFileExists{parskip.sty}{%
    \usepackage{parskip}
  }{% else
    \setlength{\parindent}{0pt}
    \setlength{\parskip}{6pt plus 2pt minus 1pt}}
}{% if KOMA class
  \KOMAoptions{parskip=half}}
\makeatother
\usepackage{xcolor}
\usepackage[top=20mm,left=20mm,heightrounded]{geometry}
\setlength{\emergencystretch}{3em} % prevent overfull lines
\setcounter{secnumdepth}{-\maxdimen} % remove section numbering
% Make \paragraph and \subparagraph free-standing
\ifx\paragraph\undefined\else
  \let\oldparagraph\paragraph
  \renewcommand{\paragraph}[1]{\oldparagraph{#1}\mbox{}}
\fi
\ifx\subparagraph\undefined\else
  \let\oldsubparagraph\subparagraph
  \renewcommand{\subparagraph}[1]{\oldsubparagraph{#1}\mbox{}}
\fi


\providecommand{\tightlist}{%
  \setlength{\itemsep}{0pt}\setlength{\parskip}{0pt}}\usepackage{longtable,booktabs,array}
\usepackage{calc} % for calculating minipage widths
% Correct order of tables after \paragraph or \subparagraph
\usepackage{etoolbox}
\makeatletter
\patchcmd\longtable{\par}{\if@noskipsec\mbox{}\fi\par}{}{}
\makeatother
% Allow footnotes in longtable head/foot
\IfFileExists{footnotehyper.sty}{\usepackage{footnotehyper}}{\usepackage{footnote}}
\makesavenoteenv{longtable}
\usepackage{graphicx}
\makeatletter
\def\maxwidth{\ifdim\Gin@nat@width>\linewidth\linewidth\else\Gin@nat@width\fi}
\def\maxheight{\ifdim\Gin@nat@height>\textheight\textheight\else\Gin@nat@height\fi}
\makeatother
% Scale images if necessary, so that they will not overflow the page
% margins by default, and it is still possible to overwrite the defaults
% using explicit options in \includegraphics[width, height, ...]{}
\setkeys{Gin}{width=\maxwidth,height=\maxheight,keepaspectratio}
% Set default figure placement to htbp
\makeatletter
\def\fps@figure{htbp}
\makeatother
% definitions for citeproc citations
\NewDocumentCommand\citeproctext{}{}
\NewDocumentCommand\citeproc{mm}{%
  \begingroup\def\citeproctext{#2}\cite{#1}\endgroup}
\makeatletter
 % allow citations to break across lines
 \let\@cite@ofmt\@firstofone
 % avoid brackets around text for \cite:
 \def\@biblabel#1{}
 \def\@cite#1#2{{#1\if@tempswa , #2\fi}}
\makeatother
\newlength{\cslhangindent}
\setlength{\cslhangindent}{1.5em}
\newlength{\csllabelwidth}
\setlength{\csllabelwidth}{3em}
\newenvironment{CSLReferences}[2] % #1 hanging-indent, #2 entry-spacing
 {\begin{list}{}{%
  \setlength{\itemindent}{0pt}
  \setlength{\leftmargin}{0pt}
  \setlength{\parsep}{0pt}
  % turn on hanging indent if param 1 is 1
  \ifodd #1
   \setlength{\leftmargin}{\cslhangindent}
   \setlength{\itemindent}{-1\cslhangindent}
  \fi
  % set entry spacing
  \setlength{\itemsep}{#2\baselineskip}}}
 {\end{list}}
\usepackage{calc}
\newcommand{\CSLBlock}[1]{\hfill\break\parbox[t]{\linewidth}{\strut\ignorespaces#1\strut}}
\newcommand{\CSLLeftMargin}[1]{\parbox[t]{\csllabelwidth}{\strut#1\strut}}
\newcommand{\CSLRightInline}[1]{\parbox[t]{\linewidth - \csllabelwidth}{\strut#1\strut}}
\newcommand{\CSLIndent}[1]{\hspace{\cslhangindent}#1}

\KOMAoption{captions}{tableheading}
\usepackage{wrapfig}
\usepackage{subcaption}
\usepackage{amsmath}
\usepackage{cancel}
\usepackage{hyperref}
\usepackage{tikz}
\usepackage{setspace}
\setstretch{1.5}
\usetikzlibrary{shapes.geometric, arrows, arrows.meta, positioning, calc}
\usepackage{tabularx}
\renewcommand{\maketitle}{}
\usepackage{fancyhdr}
\pagestyle{fancy}
\fancyhf{}
\fancyhead[L]{\rightmark}
\fancyhead[R]{\thepage}
\fancyfoot[C]{\thepage}
\usepackage{colortbl}
\definecolor{cornflowerblue}{RGB}{100,149,237}
\definecolor{darkblue}{RGB}{115,150,255}
\definecolor{lighterblue}{RGB}{131, 191, 212}
\definecolor{lightblue}{RGB}{178,211,220}
\makeatletter
\@ifpackageloaded{caption}{}{\usepackage{caption}}
\AtBeginDocument{%
\ifdefined\contentsname
  \renewcommand*\contentsname{Table of contents}
\else
  \newcommand\contentsname{Table of contents}
\fi
\ifdefined\listfigurename
  \renewcommand*\listfigurename{Figurliste}
\else
  \newcommand\listfigurename{Figurliste}
\fi
\ifdefined\listtablename
  \renewcommand*\listtablename{Tabelliste}
\else
  \newcommand\listtablename{Tabelliste}
\fi
\ifdefined\figurename
  \renewcommand*\figurename{Figur}
\else
  \newcommand\figurename{Figur}
\fi
\ifdefined\tablename
  \renewcommand*\tablename{Tabell}
\else
  \newcommand\tablename{Tabell}
\fi
}
\@ifpackageloaded{float}{}{\usepackage{float}}
\floatstyle{ruled}
\@ifundefined{c@chapter}{\newfloat{codelisting}{h}{lop}}{\newfloat{codelisting}{h}{lop}[chapter]}
\floatname{codelisting}{Listing}
\newcommand*\listoflistings{\listof{codelisting}{List of Listings}}
\makeatother
\makeatletter
\makeatother
\makeatletter
\@ifpackageloaded{caption}{}{\usepackage{caption}}
\@ifpackageloaded{subcaption}{}{\usepackage{subcaption}}
\makeatother
\ifLuaTeX
  \usepackage{selnolig}  % disable illegal ligatures
\fi
\usepackage{bookmark}

\IfFileExists{xurl.sty}{\usepackage{xurl}}{} % add URL line breaks if available
\urlstyle{same} % disable monospaced font for URLs
\hypersetup{
  colorlinks=true,
  linkcolor={blue},
  filecolor={Maroon},
  citecolor={Blue},
  urlcolor={Blue},
  pdfcreator={LaTeX via pandoc}}

\author{}
\date{}

\begin{document}


\newgeometry{left=0cm, right=0cm, top=0cm, bottom=0cm}
\vspace*{0.5cm} 
\hspace*{1.5cm}\includegraphics[width=10cm]{dokumentobjekter/texstuff/UiT_Logo_Bok_Bla_RGB.png} 


\begin{flushleft}
    \vspace*{0.5cm}
    \hspace*{2.5cm}\large{\color{black}\textbf{Formuefordeling og sykefravær}}  \\[0.5em]
\hspace*{2.5cm}\color{black}\fontsize{11}{13.2}\selectfont  Daniel Nikolai Johannessen og Daniel Fabio Groth \\[0.5em]
    \hspace*{2.5cm}{\color{black}\fontsize{11}{13.2}\selectfont Handelshøgskolen ved UiT \\[0.2em]
    \hspace*{2.5cm}\color{black}\fontsize{11}{13.2}\selectfont Juni 2025 \\[0.5em]
    \hspace*{2.0cm}
    \par}
\end{flushleft} 



\begin{tikzpicture}[remember picture, overlay]
    \node[anchor=south west, inner sep=0] at (current page.south west) {\includegraphics[width=\paperwidth]{dokumentobjekter/texstuff/forside_bilde.png}};
\end{tikzpicture}


\newgeometry{left=20mm, right=20mm, top=20mm, bottom=20mm}





\thispagestyle{plain}
\begin{center}
    \Large
    \textbf{Forord}
\end{center}


\begin{verbatim}

26. februar: Innlevering av prosjektbeskrivelse. \textbf{Gjort}

28. februar: Ferdigstille fordeling av arbeidsoppgaver \textbf{Gjort}

7. mars: Ferdigstille bakgrunn og problemstilling \textbf{Nesten ferdig.}

14. mars: Ferdigstille litteraturgjennomgang \textbf{80%}

21. mars: Begynt på teoretisk grunnlag \textbf{ja}

28. mars: Fortsatt arbeid med teoretisk grunnlag \textbf{ja}

4. april: Ferdigstille teoretisk grunnlag \textbf{25%}

11. april: Levere midtveisinnlevering og statusrapport {gjort}

18. april: Ferdigstille data og analysemetode

25. april: Begynne økonometrisk analyse 

2. mai: Sende utkast

9. mai: Ferdigstille diskusjon

16. mai: Ferdigstille konklusjon og oppgaven

23. mai: Finpussing av figur plasseringer og tabeller. referere bedre til tekst i oppgaven.

30. mai: Ferdigstille referanser og litteraturliste og gå over siste bit

1. juni: Innlevering av bacheloroppgaven

\end{verbatim}


\newpage
\hypersetup{linkcolor=black}
\renewcommand{\contentsname}{Innholdsfortegnelse}
\renewcommand*{\figureautorefname}{Figur}
\renewcommand*{\tableautorefname}{Tabell}
\tableofcontents
\listoffigures
\listoftables
\hypersetup{linkcolor=blue}
\newpage

\thispagestyle{plain}
\begin{center}
    \Large
    \textbf{Sammendrag}
\end{center}


Sammendrag her



\newpage

\section{1 Innledning}\label{innledning}

Denne bacheloroppgaven undersøker sammenhengen mellom sosioøkonomiske
forhold og sykefravær, med et spesielt fokus på hvordan endringer i
formuefordeling kan påvirke arbeidstakeres helse og fravær fra jobben.
Vi benytter en Job Demands-Resources (JD-R) modell som teoretisk
rammeverk, og analyserer data fra Levekårsundersøkelsen om arbeidsmiljø.

\paragraph{Bakgrunn}\label{bakgrunn}

I årene etter finanskrisen har vi observert en økende formueulikhet i
mange vestlige land, inkludert Norge. Denne trenden har blitt forsterket
etter pandemien, spesielt i boligmarkedet. Lønnsveksten har ikke holdt
tritt med prisøkningen på eiendeler, særlig boliger. Dette har gjort det
relativt vanskeligere for unge og de med lavere inntekter å opparbeide
seg formue, for eksempel gjennom boligkjøp. Vi antar dermed at
formuenivået til arbeidstakere har en effekt på spesielt motivasjon og
helse, at denne utviklingen kan ha negative konsekvenser for
arbeidstakeres motivasjon og helse, og dermed påvirke sykefraværet. Når
det blir stadig vanskeligere å oppnå økonomisk trygghet og en akseptabel
levestandard, kan det føre til økt stress, redusert jobbmotivasjon, og i
verste fall dårligere helse og økt fravær.

Å forstå hvordan disse endringene påvirker arbeidstakeres helse og
fravær er viktig for å kunne iverksette tiltak som kan motvirke negative
konsekvenser av økende formueulikhet. Dette kan være spesielt viktig i
en tid hvor vi ser en økende polarisering i samfunnet, og hvor det er
viktig å sikre at alle har like muligheter til å oppnå økonomisk
trygghet og god helse, uavhengig av formue og inntekt. Problemstillingen
for oppgaven er dermed: \emph{Forklarer nivået på formue sykefraværet i
Norge?}. Vi vil undersøke om forskjellige formuegrupper har ulikt
sykefravær, og om det er en sammenheng mellom formue og sykefravær. Vi
vil også se på om det er andre faktorer som påvirker sykefraværet, og om
disse faktorene kan forklare eventuelle sammenhenger mellom formue og
sykefravær. Vi vil danne oss tre hypoteser basert på teori og tidligere
forskning, og teste disse ved hjelp av en Structural Equation Model
(SEM), hvor vi kontrollerer for andre relevante faktorer, som for
eksempel alder, kjønn, utdanning og yrke.

Tidligere forskning har funnet at sosioøkonomiske forhold, som inntekt
og utdanning, har en effekt på helse og sykefravær. Jaeggi et al. (2021)
testet dette på et lite samfunn av innfødte i Tsimane i Bolivia, hvor de
fant at økt formue hadde en positiv effekt på helse, mens større ulikhet
ledet til respirasjonssykdom som førte til dødelighet. Før vi går
gjennom teori og empiri vil vi gå gjennom begrepsavklaringer, hvor vi
vil definere formue, sykefravær og andre relevante begreper. Etter
teorien vil vi gå dypere inn i tidligere forskning på temaet, og se på
hva som er funnet tidligere, og hvilke mekanismer som kan forklare
sammenhengen mellom formue og sykefravær.

\paragraph{Oppsett}\label{oppsett}

Oppgaven er delt inn i følgende kapitler: I kapittel 2 vil vi gi en
teoretisk bakgrunn for oppgaven, og gjøre rede for tidligere forskning
på temaet. I kapittel 3 vil vi forklare metode og datagrunnlag, i
kapitell 4 gjennomføres analysen og i kapittel 5 vil vi presentere
resultatene fra analysen. I kapittel 6 vil vi diskutere resultatene, og
i kapittel 7 vil vi konkludere og gi anbefalinger for videre forskning.

Avslutningsvis i appendiks har vi med relevant kode som er brukt for å
analysere dataene og en oversikt over testene som er gjort i analysen,
og til slutt en oversikt rundt bruk av kunstig intelligens i oppgaven.
\newpage

\section{2 Teori}\label{teori}

I dette kapittelet vil vi gi en teoretisk bakgrunn for oppgaven, og
gjøre rede for tidligere forskning på temaet. Vi vil først definere
begrepene kortfattet, og deretter presentere teori og empiri som er
relevant for oppgaven. Vi vil spesielt fokusere på Job Demands-Resources
(JDR) modellen, som er et mye brukt rammeverk for å forstå sammenhengen
mellom arbeidsmiljø og helse. Vi vil også se på tidligere forskning på
temaet, og se på hva som er funnet tidligere, og hvilke mekanismer som
kan forklare sammenhengen mellom formue og sykefravær.

\subsection{2.1 Job Demands-Resources (JDR)
modellen}\label{job-demands-resources-jdr-modellen}

Job Demands-Resources (JDR) modellen er et rammeverk for å forstå
hvordan arbeidsmiljøet påvirker helse og trivsel. Modellen skiller
mellom to typer faktorer: jobbkrav (job demands) og jobbressurser (job
resources). Jobbkrav refererer til kravene og utfordringene som ansatte
møter i jobben, mens jobbressurser refererer til de ressursene og
støtten som ansatte har tilgjengelig for å håndtere disse kravene.
Modellen antyder at en balanse mellom jobbkrav og jobbressurser er
viktig for å opprettholde helse og trivsel på arbeidsplassen. Høyere
jobbkrav kan føre til stress og utbrenthet, mens høyere jobbressurser
kan føre til økt motivasjon og trivsel.

https://www.wilmarschaufeli.nl/publications/Schaufeli/476.pdf

Grunnen til at vi velger JD-R modellen er fordi vi forventer at
formuenivå kan forandre jobbkrav og jobbressurser. Vi tenker også at
formuenivået har mye å si til også hvor mye jobbkrav og jobbressurser
påvirker personer.

\subsection{2.2 Hypoteser}\label{hypoteser}

Eventuelle testhypoteser

\subsection{2.3 Tidligere empirisk
forskning}\label{tidligere-empirisk-forskning}

Forklar kortfattet hva tidligere forskning har funnet generelt om
problemstillingen (hvorfor det er viktig å studere fra et
samfunnsperspektiv)

Tidligere empirisk forskning har over tid vist positive forhold mellom
forskjellige Job Demands-Resources-faktorer og årsaker som kan føre til
sykefravær.

Utbrenthet og arbeidsengasjement kan betraktes som to distinkte
psykologiske tilstander. I en empirisk studie av Schaufeli \& Bakker
(2004) ble det testet en modell hvor disse to variablene fungerte som
avhengige variabler, mens ulike Job Demands-Resources-faktorer ble
inkludert som uavhengige variabler i en Structural Equation Model (SEM).
Studien viste at utbrenthet og jobbengasjement var negativt korrelert,
og at jobbkravene hadde en positiv effekt på utbrenthet, mens
jobbressursene hadde en positiv effekt på jobbengasjement. Dette kan
tyde på at høyere jobbkrav kan føre til høyere utbrenthet, mens høyere
jobbressurser kan føre til høyere jobbengasjement.

\subsubsection{Mikro}\label{mikro}

I en annen studie av Vander Elst et al. (2016) utført på Belgisk
hjemmepleiepersonell, ble det testet en Job Demands-Resources-modell
hvor utbrenthet og jobbengasjement var utfallsvariabler. Jobbkrav og
jobbressurser ble modellert som prediktorer. Studien viste at
jobbkravene var positivt assosiert med utbrenthet, mens jobbressursene
hadde var positivt assosiert med jobbengasjement. Denne studien viser
også at høyere jobbkrav kan føre til høyere utbrenthet, mens høyere
jobbressurser kan føre til høyere jobbengasjement.

Nevnt i innledningen studerte Jaeggi et al. (2021) effekten av ulikhet i
formue i et småskala samfunn av innfødte i Tsimane i Bolivia hvor det
var 871 observasjoner med i studien, \(n = 871\). I studiet testet de
hvorvidt relativ husholdningrikdom og ulikhet i formue mot forskjellige
psykologiske variabler og helseutfall som depresjon, BMI, blodtrykk og
sykelighet. Dette ble testet mot kontrollvariabler, og studien viste til
en kobling mellom formueulikhet hvor de med lavere formue hadde større
sannsynlighet for å få høyere blodtrykk og luftveissykdommer som kunne
lede til dødsfall. De fant også at de med høyere formue hadde lavere
sannsynlighet for å få depresjon og høyere BMI. Dette kan tyde på at
ulikhet i formue kan ha en negativ effekt på helse og livskvalitet, og
vi vil videre bygge på dette i vår oppgave, for å se om det er en
sammenheng mellom formue og sykefravær i Norge, og om det er andre
faktorer som kan påvirke sykefraværet.

Langseth-Eide \& Vittersø (2021) bygger videre på tidligere forskning og
adresserer limitasjonene ved Job Demands-Resources-modellen. De
argumenterer for at Job Demands-Resources-modellen ved tidligere
forskning har hatt fokus på organisasjonsnivået, og at det er viktig å
se på hvordan Job Demands-Resources-modellen kan brukes bedre på
jobbressurser, jobbengasjement og helserelaterte utfall. De gjorde en
paneldata studie på fast ansatte i Norge med to års tidsforsinkelse med
1533 ansatte første tidsperiode, \(n =1533\) og 1503 ansatte,
\(n = 1503\) neste tidsperiode. Over lengre tid fant de at jobbressurser
hadde en positiv effekt på jobbengasjement, og at jobbengasjement var
negativt assosiert med sykefravær. Dette kan tyde på at høyere
jobbressurser kan føre til høyere jobbengasjement, som igjen kan føre
til lavere sykefravær.

I dagens samfunn er det viktig å forstå hvordan formue og ulikhet kan
påvirke helse, sykefravær og livskvalitet. Dette er spesielt relevant i
lys av den økende formueulikheten som vi har sett de siste årene, ikke
bare i Norge, men i mange vestlige land.

\subsubsection{Makro}\label{makro}

\subsection{2.4 Begrepsdefinisjoner}\label{begrepsdefinisjoner}

\paragraph{Formue}\label{formue}

Formue er et begrep som refererer til den totale verdien av eiendeler og
investeringer som en person eller husholdning eier. Dette inkluderer
kontanter, eiendom, aksjer, obligasjoner og andre finansielle eiendeler.
Formue kan også referere til nettoformue, som er forskjellen mellom
eiendeler og gjeld. Formue kan påvirke livskvalitet, helse og muligheter
for økonomisk trygghet.

\paragraph{Sykefravær}\label{sykefravuxe6r}

Sykefravær refererer til perioden en ansatt er borte fra jobb på grunn
av sykdom eller skade. Det kan være kortvarig eller langvarig, og kan
påvirkes av en rekke faktorer, inkludert helse, arbeidsmiljø og
sosioøkonomiske forhold. Sykefravær kan ha betydelige konsekvenser for
både arbeidstakere og arbeidsgivere, inkludert tap av inntekt, redusert
produktivitet og økte kostnader for helsevesenet.

\paragraph{Helse}\label{helse}

Helse refererer til en tilstand av fysisk, mentalt og sosialt velvære,
og ikke bare fravær av sykdom eller skade. Helse kan påvirkes av en
rekke faktorer, inkludert genetikk, livsstil, miljø og sosioøkonomiske
forhold. God helse er viktig for livskvalitet og trivsel, og kan påvirke
evnen til å jobbe og delta i samfunnet.

\paragraph{Sykemelding}\label{sykemelding}

I Norge i dag så dekkes sykemelding av folketrygden, og arbeidsgiver
betaler sykepenger i de første 16 dagene av sykefraværet. Etter dette
tar folketrygden over ansvaret for å betale sykepenger, og dekningen er
i dag på 100\%. Arbeidstaker har rett til full lønn i minst 3 måneder i
kalenderåret. Sykemelding kan være kortvarig eller langvarig, og kan
påvirkes av en rekke faktorer, inkludert helse, arbeidsmiljø og
sosioøkonomiske forhold.

\paragraph{Jobbkrav}\label{jobbkrav}

Jobbkrav refererer til de kravene og utfordringene som ansatte møter i
jobben. Dette kan inkludere arbeidsmengde, tidsfrister, emosjonelle krav
og fysiske krav. Jobbkrav kan påvirke helse og trivsel, og kan føre til
stress og utbrenthet hvis de er for høye eller ikke håndteres på en god
måte.

De mest vanlige jobbkravene er arbeidsmengde og tidspress. Det kan være
positive og negative jobbkrav. Positive jobbkrav kan være utfordrende og
motiverende, mens negative jobbkrav kan være overveldende og føre til
stress og utmattelse. Når dette da skjer over tid kan det føre til
sykefravær og dårligere helse.

\paragraph{Må legge til sources}\label{muxe5-legge-til-sources}

\paragraph{Jobbressurser}\label{jobbressurser}

Jobbressurser refererer til de ressursene og støtten som ansatte har
tilgjengelig for å håndtere jobbkravene. Dette kan inkludere støtte fra
kolleger og ledelse, muligheter for utvikling og læring, og
fleksibilitet i arbeidsoppgaver. Jobbressurser kan bidra til å redusere
stress og utbrenthet, og kan øke jobbengasjement og trivsel.

Jobbressurser kan være støtte fra kolleger og ledelse, muligheter for
utvikling og læring, og fleksibilitet i arbeidsoppgaver.

\paragraph{Jobbengasjement}\label{jobbengasjement}

Jobbengasjement refererer til en positiv, tilfredsstillende og energisk
tilstand av arbeidstakeren i forhold til jobben. Det kan beskrives som
en tilstand av å være fullt engasjert og involvert i arbeidet, og kan
føre til økt produktivitet, trivsel og jobbtilfredshet. Jobbengasjement
kan påvirkes av en rekke faktorer, inkludert jobbkrav, jobbressurser og
sosioøkonomiske forhold.

\paragraph{Ulikhet}\label{ulikhet}

Ulikhet refererer til forskjeller i ressurser, muligheter og livsvilkår
mellom individer eller grupper i samfunnet. Dette kan inkludere ulikhet
i inntekt, formue, utdanning og helse. Ulikhet kan påvirke livskvalitet,
helse og muligheter for økonomisk trygghet, og kan også ha negative
konsekvenser for samfunnet som helhet, inkludert økt kriminalitet,
politisk ustabilitet og redusert sosial sammenhengskraft.

\paragraph{Utbrenthet}\label{utbrenthet}

Utbrenthet refererer til en tilstand av fysisk og emosjonell utmattelse
som kan oppstå som følge av langvarig stress og belastning på jobben.
Det kan føre til redusert motivasjon, engasjement og produktivitet, samt
økt sykefravær. Utbrenthet kan påvirkes av en rekke faktorer, inkludert
arbeidsmiljø, jobbkrav og jobbressurser.

\subsection{2.5 Forklar teori og empiriske funn knyttet til koblingen
som du vil undersøke. Være nøye med å gjøre rede for
mekanismer!}\label{forklar-teori-og-empiriske-funn-knyttet-til-koblingen-som-du-vil-undersuxf8ke.-vuxe6re-nuxf8ye-med-uxe5-gjuxf8re-rede-for-mekanismer}

\subsection{2.6 Modelloppsett}\label{modelloppsett}

trur kanskje formue effekt er veldig kraftig på ung alder

\subsubsection{Dette er veldig work in
progress}\label{dette-er-veldig-work-in-progress}

Vi tenker vi også å ta å tegne opp en figur som viser hvordan modellen
fungerer i tikz. Oppsettet er veldig work in progress, og mulig vi ender
med 3 ligninger. Men planen er å lage løsninger for forskjellige nivå av
formue, for å så sette dette inn som en faktor i en nyttefunksjon slik
at vi kan tegne indifferensligninger og simplifisere.

\subsubsection{Hovedmodell for sykefravær
(SF)}\label{hovedmodell-for-sykefravuxe6r-sf}

Vi antar at sykefraværet (SF) i hovedsak påvirkes av:

Jobbkrav (JK) (effekten av arbeidsbelastning),

Motivasjon (M) (som en mekanisme/medierende faktor),

Formuenivå (FN) (som hovedprediktor og også direkte påvirker SF),

Så kan vi ha en X som er en mengde kontrollvariabler som for eksempel
avtalte dager, demografi, arbeidsrelaterte forhold osv.

\[
SF_i = \beta_0 + \beta_1 JK_i + \beta_2 M_i + \beta_3 FN_i + \Sigma_j \beta_{4j}X_{ij} + \epsilon_{1i}
\]

Der \(i\) er individet, \(\beta_0\) er konstanten, \(\beta_1\),
\(\beta_2\), \(\beta_3\) er koeffisientene for henholdsvis jobbkrav,
motivasjon og formuenivå, \(\Sigma_j \beta_{4j}X_{ij}\) fanger opp
effekter av eventuelle kontrollvariabler, \(\epsilon_{1i}\) er
feilleddet.

Denne likningen innebærer at formuenivået ikke bare antas å ha en
direkte effekt på sykefravær, men via motivasjon så kan effekten også gå
via en indirekte kanal.

\subsubsection{Ligning for motivasjon
(M)}\label{ligning-for-motivasjon-m}

Motivasjonen antas å bli påvirket av:

Jobbressurser (JR) (dvs. støtte og autonomi i arbeidet),

Formuenivå (FN) (som antas å påvirke hvor sensitiv man er for endringer
i inntekt -- dvs. hvordan man prioriterer fritid/arbeid),

Z er kontrollvariabler som f.eks. utdanning eller andre relevante
demografiske/yrkesmessige mål.

\[
M_i = \alpha_0 + \alpha_1 JR_i + \alpha_2 FN_i + \Sigma_k \alpha_{3k}Z_{ik} + \epsilon_{2i}
\]

Der \(\alpha_0\) er konstanten, \(\alpha_1\) og \(\alpha_2\) er
koeffisientene for henholdsvis jobbressurser og formuenivå,
\(\Sigma_k \alpha_{3k}Z_{ik}\) fanger opp effekter av eventuelle
kontrollvariabler, \(\epsilon_{2i}\) er feilleddet.

\newpage

\section{3 Metode og data}\label{metode-og-data}

Forklar og begrunn metodevalg i henhold til problemstillingen. Beskriv
hvordan data til oppgaven er fremskaffet. Gi en innledende oversikt over
data (deskriptiv statistikk).

I dette kapitlet går vi gjenom metode og datagrunnlag for oppgaven. Vi
vil først forklare metodevalget, og deretter beskrive datagrunnlaget og
hvordan dataene er fremskaffet. Vi vil også gi en innledende oversikt
over dataene, inkludert deskriptiv statistikk for alle variablene i
analysen.

\subsection{3.1 Metode}\label{metode}

I oppgaven vil vi bruke en kvantitativ metode for å analysere
sammenhengen mellom formue og sykefravær. Vi vil bruke en Structural
Equation Model (SEM) for å teste hypotesene våre, og vi vil kontrollere
for andre relevante faktorer som kan påvirke sykefraværet. SEM er en
statistisk metode som gjør det mulig å teste komplekse modeller med
flere variabler, og som kan håndtere både direkte og indirekte
sammenhenger mellom variablene. Vi vil bruke R for å gjennomføre
analysen, og vi vil bruke pakker som x og x for å implementere
SEM-modellen.

\subsection{3.2 Structural Equation Model
(SEM)}\label{structural-equation-model-sem}

\subsubsection{a) Ligning til modellen}\label{a-ligning-til-modellen}

\subsubsection{b) Forklaring av alle deler i
modellen}\label{b-forklaring-av-alle-deler-i-modellen}

\subsubsection{c) Beskrivning av metode}\label{c-beskrivning-av-metode}

\subsection{3.2 Data}\label{data}

\subsubsection{3.2.1 Datakilde og utvalg}\label{datakilde-og-utvalg}

\subsubsection{3.2.2 Variabler}\label{variabler}

\paragraph{a) Avhengig variabel}\label{a-avhengig-variabel}

\paragraph{b) Variabler knyttet til
hypotese}\label{b-variabler-knyttet-til-hypotese}

\paragraph{c) Kontrollvariabler}\label{c-kontrollvariabler}

\subsection{3.3 Deskriptiv statistikk}\label{deskriptiv-statistikk}

Kommentar: Deskriptiv statistikk kan evt legges til som avsnitt 5.1.3

\newpage

\section{4 Analyse}\label{analyse}

\paragraph{a) Tabell med resultat fra
regresjonsanalysen(e)}\label{a-tabell-med-resultat-fra-regresjonsanalysene}

\paragraph{b) Redegjørelse for resultat knyttet til
hypoteser}\label{b-redegjuxf8relse-for-resultat-knyttet-til-hypoteser}

\paragraph{c) Redegjørelse for effekt av
kontrollvariabler}\label{c-redegjuxf8relse-for-effekt-av-kontrollvariabler}

\paragraph{d) Redegjørelse for svakheter i
modellen/data}\label{d-redegjuxf8relse-for-svakheter-i-modellendata}

\newpage

\section{5 Resultat}\label{resultat}

Her presenteres den empiriske analysen og dens resultater. Vanligvis vil
en empirisk analyse bestå av en regresjonsanalyse med flere variabler.
Andre muligheter kan diskuteres med veilederen.

\subsection{5.1 Tabeller}\label{tabeller}

\subsection{5.2 Figurer}\label{figurer}

\subsection{5.3 Forklaring av tabeller og
figurer}\label{forklaring-av-tabeller-og-figurer}

\newpage

\section{6 Diskusjon}\label{diskusjon}

Dette kapitlet drøfter resultatene i forhold til problemstillingen. Hva
er funnet ut av, hva gjenstår, hvilke styrker og svakheter har analysen?

\subsubsection{a) Oppsummering av hva formålet med oppgaven var, og hva
analysen
viste}\label{a-oppsummering-av-hva-formuxe5let-med-oppgaven-var-og-hva-analysen-viste}

\subsubsection{b) Diskusjon av hvilke konklusjoner som kan trekkes fra
dette og om resultatene er forenlig med tidligere
funn/teori}\label{b-diskusjon-av-hvilke-konklusjoner-som-kan-trekkes-fra-dette-og-om-resultatene-er-forenlig-med-tidligere-funnteori}

\subsubsection{c) Diskusjon av svakheter i
analysen}\label{c-diskusjon-av-svakheter-i-analysen}

\subsubsection{d) Diskusjon av implikasjoner for policy gitt
svakheter}\label{d-diskusjon-av-implikasjoner-for-policy-gitt-svakheter}

\subsubsection{e) Eventuelt: diskusjon av hva framtidig forskning kan
forske videre på (basert påderes funn og svakheter i
analysen)}\label{e-eventuelt-diskusjon-av-hva-framtidig-forskning-kan-forske-videre-puxe5-basert-puxe5deres-funn-og-svakheter-i-analysen}

\newpage

\section{7 Vedlegg}\label{vedlegg}

Her legger vi til vår QMD fil.

\section{Appendiks}\label{appendiks}

\subsection{Kode}\label{kode}

\subsection{Tester}\label{tester}

\subsection{Kunstig intelligens}\label{kunstig-intelligens}

\newpage

\clearpage

\section{Referanser}\label{referanser}

\phantomsection\label{refs}
\begin{CSLReferences}{1}{0}
\bibitem[\citeproctext]{ref-jaeggi2021wealth}
Jaeggi, A. V., Blackwell, A. D., Von Rueden, C., Trumble, B. C.,
Stieglitz, J., Garcia, A. R., Kraft, T. S., Beheim, B. A., Hooper, P.
L., Kaplan, H., et al. (2021). Do wealth and inequality associate with
health in a small-scale subsistence society? \emph{Elife}, \emph{10},
e59437.

\bibitem[\citeproctext]{ref-langseth2021ticket}
Langseth-Eide, B. \& Vittersø, J. (2021). Ticket to ride: A longitudinal
journey to health and work-attendance in the jd-r model.
\emph{International Journal of Environmental Research and Public
Health}, \emph{18}(8), 4327.

\bibitem[\citeproctext]{ref-schaufeli2004job}
Schaufeli, W. B. \& Bakker, A. B. (2004). Job demands, job resources,
and their relationship with burnout and engagement: A multi-sample
study. \emph{Journal of Organizational Behavior: The International
Journal of Industrial, Occupational and Organizational Psychology and
Behavior}, \emph{25}(3), 293--315.

\bibitem[\citeproctext]{ref-vander2016job}
Vander Elst, T., Cavents, C., Daneels, K., Johannik, K., Baillien, E.,
Van den Broeck, A. \& Godderis, L. (2016). Job demands--resources
predicting burnout and work engagement among belgian home health care
nurses: A cross-sectional study. \emph{Nursing Outlook}, \emph{64}(6),
542--556.

\end{CSLReferences}

\clearpage

\appendix

\section {Appendix Generell KI bruk}

I løpet av koden så kan det ses mange \# kommentarer der det er skrevet
for eks ``\#fillbetween q1 and q2''. Når vi skriver kode i Visual Studio
Code og Rstudio så er det en plugin som heter Github Copilot. Når vi
skriver slike kommentarer eller bare skriver kode så kan den foresøke å
fullføre kodelinjene mens vi skriver de. Noen ganger klarer den det, men
andre ikke. Det er vanskelig å dokumentere hvert bruk der den er brukt
siden det ``går veldig fort'' men siden vi ikke har fått på plass en
slik dokumentasjon så kan all kode der det er brukt kommentarer antas
som at det er brukt Github Copilot. Nærmere info om dette KI verktøyet
kan ses på \url{https://github.com/features/copilot}

\clearpage

\section{notater}\label{notater}

er det avvik mellom fastsatt arbeidstid og hvor mye folk arbeider?

Er folk overarbeidet?

Hvordan har antall legebesøk endret seg samtidig som legemeldt
sykefravær har endret seg. er leger overarbeidet og skriver ut for mye
sykefravær?

Er sykefraværet et problem? Hvordan har sysselsettingsrate endret seg
med sykefravær? er det spesiell korrelasjon mellom egenmeldt og
legemeldt der?

Dårlig ledelse og lite engasjerte arbeidere?

https://www.dagensperspektiv.no/synspunkt/benedicte-langseth-eide-svarer-hr-norge-om-sykefravaer-og-ledelse/1262876

https://www.nord24.no/nar-bedriftene-sliter-med-hoyt-sykefravar-ringer-de-benedicte-disse-tiltakene-nytter/s/5-32-197683

https://www.mdpi.com/1660-4601/18/8/4327

The results provide longitudinal evidence that two well-established job
resources (i.e., social support and feedback) predicted work engagement,
that work engagement was negatively related to sick leave and that this
relation was mediated by subjective health. By showing that
health-related indicators could also be outcomes of the motivational
process in the JD-R model, we have strengthened the model.

https://munin.uit.no/handle/10037/15801

The results also revealed that both workaholics and work-engaged
employees put in more hours at work than was expected of them. We found
that workaholism was negatively related to work-related health, whereas
work engagement was positively related to work-related health. These
findings support the notion of workaholism and work engagement as two
different forms of working hard.

Kanskje en form for ``intensitet'' i hvor sensitiv du er.

Jeg tror formue spiller inn til hvor sensitiv du er til endringer i
inntekt. Altså ditt konsumnnivå eller etterspurt fritid endrer seg ulikt
basert på om du har mye formue eller ikke. Dette kan være fordi du har
mer buffer til å tåle endringer i inntekt.

trur vi blir å få noe bue på den effekten. fattige, vanlige, rike,
megarike vil ha ulik effekt av motivasjon og sånt. e du megarik så har
det jo ingenting og si, e du syk eller vil ta fri så blir du hjemme, men
samtidig så vil du kanskje være spesielt sensitiv om du e fattig og at
om du da e syk eller vil ta fri så vil du både ha dårligere utgangspunkt
i jobbtype og sånt, og også kunne rett å slett være mer syk

mens de i midten rundt ``vanlige'' mot bare rike kan ha 0 effekt, men
kommer vel an på kor mye man ska mene formue har å si til hvor sensitiv
du er til endringer eller potentielle endringer i inntekt derfor æ
tenkte å bare ha det til å være en funksjon av formue kunne være enklere
motivasjon og sånt altså både på bunn og på topp så vil du også ha økt
den stygge m'en ved at du får statlige overføringe som fattig men mye
kapitalfortjeneste som rik

så formue har effekt på hvor mye utdanning du har. formue har effekt på
hvilken motivasjon du har. formue har effekt på m som er annen inntekt
utenom jobb.

g = formue, j = alder, k = utdanning, l = motivasjon

\begin{itemize}
\tightlist
\item
  v = dummyvariabel
\end{itemize}

\[
t^a = h^* - \alpha w - \beta(m(g) + h^*w) - (k\cdot v+j\cdot v)
\]

Dummy variabler for ulike aldersgrupper. beholde en ligning for alle men
da bruke de dummyvariablene. dermed kunne tolke bare en variabel.

forskjellige typer inntekt påvirke forskjellig i m variabelen.

Grunn til cb er at den er enkel og at vi nesten alltid tar log av
dataen. om vi har 0 variabler så blir det bare tull.

\subsection{Notater}\label{notater-1}

\subsubsection{Inntektsfattigdom og
levekårsfattigdom}\label{inntektsfattigdom-og-levekuxe5rsfattigdom}

https://www.ssb.no/sosiale-forhold-og-kriminalitet/artikler-og-publikasjoner/inntektsfattig-eller-levekaarsfattig

Hva så med en mer absolutt tilnærming i form av et forbruksbudsjett
inkludert faktiske bokostnader? Det enkleste målet som ikke tar hensyn
til verken studenter eller formuende, har omtrent like sterk sammenheng
med levekårsfattigdom som den vi finner ved EU60, og dermed noe sterkere
enn ved OECD50. Ved å holde studenter og/eller formuende utenfor
definisjonen med budsjettilnærming, får vi de samme virkningene som vi
har sett tidligere. Det å holde formuende utenfor bidrar til sterkere
sammenheng med levekårsfattigdom, mens det å holde studenter utenfor
ikke gjør det.

Våre funn viser dermed at det ikke er avgjørende om vi definerer
inntektsfattigdom absolutt (ved bruk av husholdningsbudsjett) eller
relativt (ved bruk av ekvivalensskala og inntektsfordeling) når vi ser
på sammenhengen med levekårsfattigdom. Den viktigste faktoren synes å
være at vi tar hensyn til formue, som er en buffer mot mange av
levekårsproblemene. Det har imidlertid ikke særlig betydning å ta hensyn
til studenter i denne sammenhengen, selv om det bidrar til å redusere
andelen inntektsfattige

\subsubsection{Helse og formue}\label{helse-og-formue}

https://pmc.ncbi.nlm.nih.gov/articles/PMC8225390/

controlling for community-average wealth, age, sex, household size,
community size, and distance to markets. Wealthier people largely had
better outcomes while inequality associated with more respiratory
disease, a leading cause of mortality. Greater inequality and lower
wealth were associated with higher blood pressure. Psychosocial factors
did not mediate wealth-health associations. Thus, relative
socio-economic position and inequality may affect health across diverse
societies, though this is likely exacerbated in high-income countries.

\subsubsection{Gatsby curve}\label{gatsby-curve}

``great gatsby curve'' med vedvarende inntekt på tvers av generasjoner.
og siden fattige ikke blir spesielt mye fattigere enn middelklassen, men
at det heller er rikere som flyr fremover. -\textgreater{} kanskje
større forskjell på median og gjennomsnittlig inntekt/formue altså flere
som er ikke rike som jobber i mer sånn lav inntekt yrker og barnehager å
sånt me mye sykdom

\subsubsection{karriærevalg, utdanning
osv.}\label{karriuxe6revalg-utdanning-osv.}

fattigere har dårligere tilgang på ``career role models'' som gjør at de
kanskje ikke vet om de bedre yrkene og sånt og dermed igjen blir mindre
utdanna og sånt
https://www.gallup.com/analytics/506696/amazon-research-hub.aspx

\subsubsection{Stress? glemte studie
her}\label{stress-glemte-studie-her}

inntektsusikkerhet -\textgreater{} økt stress

\subsubsection{Motivasjonseffekt av
ulikhet}\label{motivasjonseffekt-av-ulikhet}

``The motivational cost of inequality: Opportunity gaps reduce the
willingness to work'' https://pmc.ncbi.nlm.nih.gov/articles/PMC7473543/

https://www.brookings.edu/articles/income-inequality-social-mobility-and-the-decision-to-drop-out-of-high-school/

ulikhet gjør at fattige blir mindre motivert siden dem føler det å bli
rik er ``umulig'' og dermed investerer mindre i seg -\textgreater{}
lavere motivasjon og lavere utdanning. kanskje mer fysisk arbeid.

\subsection{Notater}\label{notater-2}

Har høy/lav formue effekt på motivasjonen fra lønnen til arbeid. lav
formue + høy lønn = høy motivasjon? høy formue + høy lønn = ``bryr meg
ikke'' = høy formue+ lav lønn, lav formue + lav lønn = lav motivajon

Kapitalinntekter som rente/aksje osv i forhold til bruttofinanskapital i
alt. kan det være at de med høy formue utenom bolig da har mer andre
inntekter, eller at høy formue bare er lik bolig for mange.

\subsubsection{Formueeffekt på konsum}\label{formueeffekt-puxe5-konsum}

https://fnce.wharton.upenn.edu/wp-content/uploads/2019/08/chodorowreich-crns\_stock\_wealth\_effects.pdf

for hver dollar i formue du har så har du 0.028usd mer i konsum eller
noe

https://usa.visa.com/partner-with-us/visa-consulting-analytics/economic-insights/the-sudden-increase-in-the-wealth-effect-and-its-impact-on-spending.html

så vi kan vise til hvordan de med lav formue da kan være tvungen til å
ta mer tima selv med lav motivasjon for samme konsumnivå fant det
tilfeldigvis her
https://www.economist.com/finance-and-economics/2025/03/19/the-trump-administration-is-playing-a-dangerous-stockmarket-game



\end{document}
