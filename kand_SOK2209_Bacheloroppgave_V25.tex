% Options for packages loaded elsewhere
\PassOptionsToPackage{unicode}{hyperref}
\PassOptionsToPackage{hyphens}{url}
\PassOptionsToPackage{dvipsnames,svgnames,x11names}{xcolor}
%
\documentclass[
  12pt,
  a4paper,
  DIV=11,
  numbers=noendperiod]{scrartcl}

\usepackage{amsmath,amssymb}
\usepackage{iftex}
\ifPDFTeX
  \usepackage[T1]{fontenc}
  \usepackage[utf8]{inputenc}
  \usepackage{textcomp} % provide euro and other symbols
\else % if luatex or xetex
  \usepackage{unicode-math}
  \defaultfontfeatures{Scale=MatchLowercase}
  \defaultfontfeatures[\rmfamily]{Ligatures=TeX,Scale=1}
\fi
\usepackage{lmodern}
\ifPDFTeX\else  
    % xetex/luatex font selection
  \setmainfont[]{Times New Roman}
\fi
% Use upquote if available, for straight quotes in verbatim environments
\IfFileExists{upquote.sty}{\usepackage{upquote}}{}
\IfFileExists{microtype.sty}{% use microtype if available
  \usepackage[]{microtype}
  \UseMicrotypeSet[protrusion]{basicmath} % disable protrusion for tt fonts
}{}
\makeatletter
\@ifundefined{KOMAClassName}{% if non-KOMA class
  \IfFileExists{parskip.sty}{%
    \usepackage{parskip}
  }{% else
    \setlength{\parindent}{0pt}
    \setlength{\parskip}{6pt plus 2pt minus 1pt}}
}{% if KOMA class
  \KOMAoptions{parskip=half}}
\makeatother
\usepackage{xcolor}
\usepackage[top=20mm,left=20mm,heightrounded]{geometry}
\setlength{\emergencystretch}{3em} % prevent overfull lines
\setcounter{secnumdepth}{5}
% Make \paragraph and \subparagraph free-standing
\ifx\paragraph\undefined\else
  \let\oldparagraph\paragraph
  \renewcommand{\paragraph}[1]{\oldparagraph{#1}\mbox{}}
\fi
\ifx\subparagraph\undefined\else
  \let\oldsubparagraph\subparagraph
  \renewcommand{\subparagraph}[1]{\oldsubparagraph{#1}\mbox{}}
\fi


\providecommand{\tightlist}{%
  \setlength{\itemsep}{0pt}\setlength{\parskip}{0pt}}\usepackage{longtable,booktabs,array}
\usepackage{calc} % for calculating minipage widths
% Correct order of tables after \paragraph or \subparagraph
\usepackage{etoolbox}
\makeatletter
\patchcmd\longtable{\par}{\if@noskipsec\mbox{}\fi\par}{}{}
\makeatother
% Allow footnotes in longtable head/foot
\IfFileExists{footnotehyper.sty}{\usepackage{footnotehyper}}{\usepackage{footnote}}
\makesavenoteenv{longtable}
\usepackage{graphicx}
\makeatletter
\def\maxwidth{\ifdim\Gin@nat@width>\linewidth\linewidth\else\Gin@nat@width\fi}
\def\maxheight{\ifdim\Gin@nat@height>\textheight\textheight\else\Gin@nat@height\fi}
\makeatother
% Scale images if necessary, so that they will not overflow the page
% margins by default, and it is still possible to overwrite the defaults
% using explicit options in \includegraphics[width, height, ...]{}
\setkeys{Gin}{width=\maxwidth,height=\maxheight,keepaspectratio}
% Set default figure placement to htbp
\makeatletter
\def\fps@figure{htbp}
\makeatother
% definitions for citeproc citations
\NewDocumentCommand\citeproctext{}{}
\NewDocumentCommand\citeproc{mm}{%
  \begingroup\def\citeproctext{#2}\cite{#1}\endgroup}
\makeatletter
 % allow citations to break across lines
 \let\@cite@ofmt\@firstofone
 % avoid brackets around text for \cite:
 \def\@biblabel#1{}
 \def\@cite#1#2{{#1\if@tempswa , #2\fi}}
\makeatother
\newlength{\cslhangindent}
\setlength{\cslhangindent}{1.5em}
\newlength{\csllabelwidth}
\setlength{\csllabelwidth}{3em}
\newenvironment{CSLReferences}[2] % #1 hanging-indent, #2 entry-spacing
 {\begin{list}{}{%
  \setlength{\itemindent}{0pt}
  \setlength{\leftmargin}{0pt}
  \setlength{\parsep}{0pt}
  % turn on hanging indent if param 1 is 1
  \ifodd #1
   \setlength{\leftmargin}{\cslhangindent}
   \setlength{\itemindent}{-1\cslhangindent}
  \fi
  % set entry spacing
  \setlength{\itemsep}{#2\baselineskip}}}
 {\end{list}}
\usepackage{calc}
\newcommand{\CSLBlock}[1]{\hfill\break\parbox[t]{\linewidth}{\strut\ignorespaces#1\strut}}
\newcommand{\CSLLeftMargin}[1]{\parbox[t]{\csllabelwidth}{\strut#1\strut}}
\newcommand{\CSLRightInline}[1]{\parbox[t]{\linewidth - \csllabelwidth}{\strut#1\strut}}
\newcommand{\CSLIndent}[1]{\hspace{\cslhangindent}#1}

\KOMAoption{captions}{tableheading}
\usepackage{wrapfig}
\usepackage{subcaption}
\usepackage{amsmath}
\usepackage{cancel}
\usepackage{hyperref}
\usepackage{tikz}
\usepackage{setspace}
\setstretch{1.5}
\usetikzlibrary{shapes.geometric, arrows, arrows.meta, positioning, calc}
\usepackage{tabularx}
\renewcommand{\maketitle}{}
\usepackage{fancyhdr}
\pagestyle{fancy}
\fancyhf{}
\fancyhead[L]{\rightmark}
\fancyhead[R]{\thepage}
\fancyfoot[C]{\thepage}
\usepackage{colortbl}
\definecolor{cornflowerblue}{RGB}{100,149,237}
\definecolor{darkblue}{RGB}{115,150,255}
\definecolor{lighterblue}{RGB}{131, 191, 212}
\definecolor{lightblue}{RGB}{178,211,220}
\makeatletter
\@ifpackageloaded{caption}{}{\usepackage{caption}}
\AtBeginDocument{%
\ifdefined\contentsname
  \renewcommand*\contentsname{Table of contents}
\else
  \newcommand\contentsname{Table of contents}
\fi
\ifdefined\listfigurename
  \renewcommand*\listfigurename{Figurliste}
\else
  \newcommand\listfigurename{Figurliste}
\fi
\ifdefined\listtablename
  \renewcommand*\listtablename{Tabelliste}
\else
  \newcommand\listtablename{Tabelliste}
\fi
\ifdefined\figurename
  \renewcommand*\figurename{Figur}
\else
  \newcommand\figurename{Figur}
\fi
\ifdefined\tablename
  \renewcommand*\tablename{Tabell}
\else
  \newcommand\tablename{Tabell}
\fi
}
\@ifpackageloaded{float}{}{\usepackage{float}}
\floatstyle{ruled}
\@ifundefined{c@chapter}{\newfloat{codelisting}{h}{lop}}{\newfloat{codelisting}{h}{lop}[chapter]}
\floatname{codelisting}{Listing}
\newcommand*\listoflistings{\listof{codelisting}{List of Listings}}
\makeatother
\makeatletter
\makeatother
\makeatletter
\@ifpackageloaded{caption}{}{\usepackage{caption}}
\@ifpackageloaded{subcaption}{}{\usepackage{subcaption}}
\makeatother
\ifLuaTeX
  \usepackage{selnolig}  % disable illegal ligatures
\fi
\usepackage{bookmark}

\IfFileExists{xurl.sty}{\usepackage{xurl}}{} % add URL line breaks if available
\urlstyle{same} % disable monospaced font for URLs
\hypersetup{
  colorlinks=true,
  linkcolor={blue},
  filecolor={Maroon},
  citecolor={Blue},
  urlcolor={Blue},
  pdfcreator={LaTeX via pandoc}}

\author{}
\date{}

\begin{document}


\newgeometry{left=0cm, right=0cm, top=0cm, bottom=0cm}
\vspace*{0.5cm} 
\hspace*{1.5cm}\includegraphics[width=10cm]{dokumentobjekter/texstuff/UiT_Logo_Bok_Bla_RGB.png} 


\begin{flushleft}
    \vspace*{0.5cm}
    \hspace*{2.5cm}\large{\color{black}\textbf{Formuefordeling og sykefravær}}  \\[0.5em]
\hspace*{2.5cm}\color{black}\fontsize{11}{13.2}\selectfont  Daniel Nikolai Johannessen og Daniel Fabio Groth \\[0.5em]
    \hspace*{2.5cm}{\color{black}\fontsize{11}{13.2}\selectfont Handelshøgskolen ved UiT \\[0.2em]
    \hspace*{2.5cm}\color{black}\fontsize{11}{13.2}\selectfont Juni 2025 \\[0.5em]
    \hspace*{2.0cm}
    \par}
\end{flushleft} 



\begin{tikzpicture}[remember picture, overlay]
    \node[anchor=south west, inner sep=0] at (current page.south west) {\includegraphics[width=\paperwidth]{dokumentobjekter/texstuff/forside_bilde.png}};
\end{tikzpicture}


\newgeometry{left=20mm, right=20mm, top=20mm, bottom=20mm}





\thispagestyle{plain}
\begin{center}
    \Large
    \textbf{Forord}
\end{center}


Vi vil takke vår veileder Espen Sirnes for strålende veiledning og flotte samtaler på kontoret og Derek J. Clark for ekstra tilbakemeldinger under skriving.

\newpage
\hypersetup{linkcolor=black}
\renewcommand{\contentsname}{Innholdsfortegnelse}
\renewcommand*{\figureautorefname}{Figur}
\renewcommand*{\tableautorefname}{Tabell}
\tableofcontents
\listoffigures
\listoftables
\hypersetup{linkcolor=blue}
\newpage

\thispagestyle{plain}
\begin{center}
    \Large
    \textbf{Sammendrag}
\end{center}


Sammendrag her



\newpage

\section{Innledning}\label{innledning}

Denne bacheloroppgaven undersøker sammenhengen mellom sosioøkonomiske
forhold og sykefravær, med et spesielt fokus på hvordan endringer i
formuefordeling kan påvirke arbeidstakeres helse og fravær fra jobben.
Vi benytter en Job Demands-Resources (JD-R) modell som teoretisk
rammeverk, og analyserer data fra Levekårsundersøkelsen om arbeidsmiljø.

\subsubsection{Bakgrunn}\label{bakgrunn}

I årene etter finanskrisen har vi observert en økende formueulikhet i
mange vestlige land, inkludert Norge. Denne trenden har blitt forsterket
etter pandemien, spesielt i boligmarkedet, hvor vi ser at lønnsveksten
ikke har holdt tritt med prisøkningen på eiendeler. Dette har gjort det
relativt vanskeligere for unge og de med lavere inntekter å opparbeide
seg formue, for eksempel gjennom boligkjøp. Vi forventer dermed at
formuenivået til arbeidstakere har en effekt på spesielt motivasjon og
helse, og dermed påvirke sykefraværet. Når det blir stadig vanskeligere
å oppnå økonomisk trygghet og en akseptabel levestandard, kan det føre
til økt stress, redusert jobbmotivasjon, og i verste fall dårligere
helse og økt fravær. Hypotesene våre er basert på Job Demands-Resources
(JDR) modellen, som antyder at jobbkrav og jobbressurser påvirker
sykefravær, og at formue kan moderere disse effektene. Hovedsakelig vil
vi se på hvordan formue påvirker sykefravær, og der forventer vi at
høyere formue gir lavere sykefravær og at høyere formue demper de
negative effektene av jobbkrav og forsterker de positive effektene av
jobbressurser. Se kapittel \ref{sec-hypot} for full oversikt.

Å forstå hvordan disse endringene påvirker arbeidstakeres helse og
fravær er viktig for å kunne iverksette tiltak som kan motvirke negative
konsekvenser av økende formueulikhet. Dette kan være spesielt viktig i
en tid hvor vi ser en økende polarisering i samfunnet, og hvor det er
viktig å sikre at alle har like muligheter til å oppnå økonomisk
trygghet og god helse, uavhengig av formue og inntekt. Problemstillingen
for oppgaven er dermed: \emph{Forklarer nivået på formue sykefraværet i
Norge?}. Vi vil undersøke om forskjellige formuegrupper har ulikt
sykefravær, og om det er en sammenheng mellom formue og sykefravær. Vi
vil også se på om det er andre faktorer som påvirker sykefraværet, og om
disse faktorene kan forklare eventuelle sammenhenger mellom formue og
sykefravær. Vi vil danne oss tre hypoteser basert på teori og tidligere
forskning, og teste disse ved hjelp av en Structural Equation Model
(SEM), hvor vi kontrollerer for andre relevante faktorer, som for
eksempel alder, kjønn, utdanning og yrke.

Tidligere forskning har funnet at sosioøkonomiske forhold, som inntekt
og utdanning, har en effekt på helse og sykefravær. Jaeggi et al. (2021)
testet dette på et lite samfunn av innfødte i Tsimane i Bolivia, hvor de
fant at økt formue hadde en positiv effekt på helse, mens større ulikhet
ledet til respirasjonssykdom som økte dødeligheten. Før vi går gjennom
teori og empiri vil vi gå gjennom begrepsavklaringer, hvor vi vil
definere formue, sykefravær og andre relevante begreper. Etter teorien
vil vi gå dypere inn i tidligere forskning på temaet, og se på hva som
er funnet tidligere, og hvilke mekanismer som kan forklare sammenhengen
mellom formue og sykefravær.

\subsubsection{Oppsett}\label{oppsett}

Oppgaven er delt inn i følgende kapitler: I kapittel 2 vil vi gi en
teoretisk bakgrunn for oppgaven, og gjøre rede for tidligere forskning
på temaet. I kapittel 3 vil vi forklare metode og datagrunnlag, i
kapitell 4 gjennomføres analysen og i kapittel 5 vil vi presentere
resultatene fra analysen. I kapittel 6 vil vi diskutere resultatene, og
i kapittel 7 vil vi konkludere og gi anbefalinger for videre forskning.

Avslutningsvis i appendiks har vi med relevant kode som er brukt for å
analysere dataene og en oversikt over testene som er gjort i analysen,
og til slutt en oversikt rundt bruk av kunstig intelligens i oppgaven.

\newpage

\section{Teori}\label{teori}

I dette kapittelet vil vi gi en teoretisk bakgrunn for oppgaven, og
gjøre rede for tidligere forskning på temaet. Vi vil først definere
begrepene kortfattet, og deretter presentere teori og empiri som er
relevant for oppgaven. Vi vil spesielt fokusere på Job Demands-Resources
(JDR) modellen, som er et mye brukt rammeverk for å forstå sammenhengen
mellom arbeidsmiljø og helse. Vi vil også se på tidligere forskning på
temaet, og se på hva som er funnet tidligere, og hvilke mekanismer som
kan forklare sammenhengen mellom formue og sykefravær.

\subsection{Begrepsdefinisjoner}\label{begrepsdefinisjoner}

\subsubsection{Formue}\label{formue}

Formue er et begrep som refererer til den totale verdien av eiendeler og
investeringer som en person eller husholdning eier. Dette inkluderer
kontanter, eiendom, aksjer, obligasjoner og andre finansielle eiendeler.
Formue kan også referere til nettoformue, som er forskjellen mellom
eiendeler og gjeld.

I studien vår vil vi bruke variabelen bruttofinanskapital som en proxy
for formue. Bruttofinanskapital omfatter bankinnskudd, andeler i
aksje-,obligasjons- og pengefond, aksjer og obligasjons- og
pengemarkedsfond, formue i aksjesparekonto, obligasjoner, aksjer og
andre verdipapirer per definisjon fra
\href{https://www.ssb.no/a/metadata/conceptvariable/vardok/3449/nb}{SSB}.
(SSB, 2017) Vi blir å bruke formue som en forventet moderator i vår
analyse, og vil se hvordan formue påvirker sykefraværet.

\subsubsection{Sykefravær}\label{sykefravuxe6r}

Sykefravær refererer til perioden en ansatt er borte fra jobb på grunn
av sykdom eller skade dokumentert med egenmelding eller legemelding, i
henhold til norske lover og avtaler per definisjon fra
\href{https://www.ssb.no/arbeid-og-lonn/arbeidsmiljo-sykefravaer-og-arbeidskonflikter/statistikk/sykefravaer}{SSB}.
(SSB,2025)

I vår analyse vil vi bruke sykefraværsprosenten som avhengig variabel.
Sykefraværsprosenten er definert som antall sykefraværsdager i prosent
av totalt antall arbeidsdager i en gitt periode:

\[
SF = \frac{ \text{Antall sykefraværsdager}}{\text{Antall avtalte dagsverk}}  \times 100
\]

\subsubsection{Helse}\label{helse}

Helse refererer til en tilstand av fysisk, mentalt og sosialt velvære,
og ikke bare fravær av sykdom eller skade. Helse kan påvirkes av en
rekke faktorer, inkludert genetikk, livsstil, miljø og sosioøkonomiske
forhold. God helse er viktig for livskvalitet og trivsel, og kan påvirke
evnen til å jobbe og delta i samfunnet.

\subsubsection{Sykemelding}\label{sykemelding}

I Norge i dag så dekkes sykemelding av folketrygden, og arbeidsgiver
betaler sykepenger i de første 16 dagene av sykefraværet. Etter dette
tar folketrygden over ansvaret for å betale sykepenger, og dekningen er
i dag på 100\%. Arbeidstaker har rett til full lønn i minst 3 måneder i
kalenderåret. Sykemelding kan være kortvarig eller langvarig, og kan
påvirkes av en rekke faktorer, inkludert helse, arbeidsmiljø og
sosioøkonomiske forhold.

\subsubsection{Jobbkrav}\label{jobbkrav}

Jobbkrav refererer til de kravene og utfordringene som ansatte må gjøre
i jobben. Mer spesifikt, så refereres det til de fysiske, psykologiske,
sosiale og organisatoriske kravene som stilles til ansatte i løpet av
arbeidsdagen, og som derfor assosieres med fysiologiske eller
psykologiske kostnader.(Schaufeli \& Bakker, 2004)

Jobbkrav kan være både fysiske og psykiske, og kan inkludere krav som
arbeidsmengde, tidsfrister, ansvar, og emosjonelle krav. Jobbkrav kan
føre til stress og utbrenthet, og kan påvirke jobbengasjement og trivsel
negativt.

I vår analyse vil vi gjøre jobbkrav om til en latent variabel som består
av flere observerbare variabler. I denne variabelen vil vi inkludere
variabler som måler arbeidsmengde, arbeidstempo og hvor mye ekstra
arbeid som kreves i jobb.

\subsubsection{Jobbressurser}\label{jobbressurser}

Jobbressurser refererer til de fysiske og psykologiske, sosiale eller
organisatoriske aspektene ved jobben som bidrar til å redusere jobbkrav
og de assosierte psykologiske og fysiologiske kostnadene. Jobbressurser
kan også bidra til å oppnå arbeidsmål, fremme personlig vekst og
utvikling, og øke jobbengasjement og trivsel. Jobbressurser kan være
både interne og eksterne, og kan inkludere faktorer som støtte fra
kolleger og ledelse, muligheter for utvikling og læring, autonomi i
arbeidet, og fleksibilitet i arbeidsoppgaver. (Schaufeli \& Bakker,
2004)

Vi blir å bruke jobbressurser som en latent variabel som består av
følgende observerbare variabler: støtte fra sjef, støtte fra kolleger,
tilbakemelding fra sjef, arbeidsresultater, grad av selvbestemmelse i
oppgaver og arbeid som skal gjøres, grad av arbeidstempo og grad av
påvirkning på beslutninger i arbeidet.

\subsubsection{Jobbengasjement}\label{jobbengasjement}

Jobbengasjement refererer til en positiv, tilfredsstillende og energisk
tilstand av arbeidstakeren i forhold til jobben. Det kan beskrives som
en tilstand av å være fullt engasjert og involvert i arbeidet, og kan
føre til økt produktivitet, trivsel og jobbtilfredshet. Jobbengasjement
kan påvirkes av en rekke faktorer, inkludert jobbkrav, jobbressurser og
sosioøkonomiske forhold.

\subsubsection{Ulikhet}\label{ulikhet}

Ulikhet refererer til forskjeller i ressurser, muligheter og livsvilkår
mellom individer eller grupper i samfunnet. Dette kan inkludere ulikhet
i inntekt, formue, utdanning og helse. Ulikhet kan påvirke livskvalitet,
helse og muligheter for økonomisk trygghet, og kan også ha negative
konsekvenser for samfunnet som helhet, inkludert økt kriminalitet,
politisk ustabilitet og redusert sosial sammenhengskraft.

\subsubsection{Utbrenthet}\label{utbrenthet}

Utbrenthet refererer til en tilstand av fysisk og emosjonell utmattelse
som kan oppstå som følge av langvarig stress og belastning på jobben.
Det kan føre til redusert motivasjon, engasjement og produktivitet, samt
økt sykefravær. Utbrenthet kan påvirkes av en rekke faktorer, inkludert
arbeidsmiljø, jobbkrav og jobbressurser.

\subsection{Job Demands-Resources (JDR)
modellen}\label{job-demands-resources-jdr-modellen}

Job Demands-Resources-modellen (Schaufeli \& Bakker, 2004) er et
rammeverk for å forstå hvordan arbeidsmiljøet påvirker helse og trivsel.
Modellen skiller mellom to typer faktorer: jobbkrav (job demands) og
jobbressurser (job resources). Jobbkrav refererer til kravene og
utfordringene som ansatte møter i jobben, mens jobbressurser refererer
til de ressursene og støtten som ansatte har tilgjengelig for å håndtere
disse kravene. Modellen antyder at en balanse mellom jobbkrav og
jobbressurser er viktig for å opprettholde helse og trivsel på
arbeidsplassen. Høyere jobbkrav kan føre til stress og utbrenthet, mens
høyere jobbressurser kan føre til økt motivasjon og trivsel.

Grunnen til at vi velger JD-R modellen er fordi vi forventer at
formuenivå kan forandre jobbkrav og jobbressurser. Vi tenker også at
formuenivået har mye å si til også hvor mye jobbkrav og jobbressurser
påvirker personer.

\subsubsection{Formue i JD-R}\label{formue-i-jd-r}

Vi mener at økonomiske ressurser som formue, kan hjelpe med å forklare
sykefraværet enda mer og vil bruke den som en ekstern modererende
faktor.

Formue gir en økonomisk buffer som kan redusere sårbarheten for
jobbrelatert stress. Personer med høy formue kan ha større valgfrihet i
arbeidslivet, og tåler lettere perioder med høy belastning uten at det
går like hardt utover helse eller jobbmotivasjon. Men personer med lav
eller negativ formue vil ofte være mer økonomisk avhengige av inntekten
fra arbeid, og kan derfor være mer sårbare for jobbrelatert stress

Formue kan også ha betydning for fremtidsperspektiv og indre motivasjon.
Personer med lav formue kan oppleve mindre kontroll over egen
livssituasjon og lavere forventninger til fremtidig økonomisk trygghet,
noe som potensielt svekker arbeidsglede og motivasjon.

Vi antar da at formue påvirker hvordan individet opplever og håndterer
jobbkrav og jobbressurser, og at det har indirekte effekter via
motivasjon som påvirker sykefravær.

Ved å inkludere formue som en ekstern faktor i JD-R modellen, forsøker
vi å fange både den direkte effekten av økonomisk trygghet og hvordan
denne tryggheten forsterker eller demper effektene av jobbrelaterte
faktorer. I et samfunn med økende økonomisk ulikheter hvor forskjellen
mellom dem som har og dem som ikke har, blir større og større, er det
viktig å forstå hvordan dette påvirker arbeidstakere og deres helse.

\subsection{Tidligere empirisk
forskning}\label{tidligere-empirisk-forskning}

Forklar kortfattet hva tidligere forskning har funnet generelt om
problemstillingen (hvorfor det er viktig å studere fra et
samfunnsperspektiv)

Tidligere empirisk forskning har over tid vist positive forhold mellom
forskjellige Job Demands-Resources-faktorer og årsaker som kan føre til
sykefravær.

Utbrenthet og arbeidsengasjement kan betraktes som to distinkte
psykologiske tilstander. I en empirisk studie av Schaufeli \& Bakker
(2004) ble det testet en modell hvor disse to variablene fungerte som
avhengige variabler, mens ulike Job Demands-Resources-faktorer ble
inkludert som uavhengige variabler i en Structural Equation Model (SEM).
Studien viste at utbrenthet og jobbengasjement var negativt korrelert,
og at jobbkravene hadde en positiv effekt på utbrenthet, mens
jobbressursene hadde en positiv effekt på jobbengasjement. Dette kan
tyde på at høyere jobbkrav kan føre til høyere utbrenthet, mens høyere
jobbressurser kan føre til høyere jobbengasjement.

\subsubsection{Mikro}\label{mikro}

I en annen studie av Vander Elst et al. (2016) utført på Belgisk
hjemmepleiepersonell, ble det testet en Job Demands-Resources-modell
hvor utbrenthet og jobbengasjement var utfallsvariabler. Jobbkrav og
jobbressurser ble modellert som prediktorer. Studien viste at
jobbkravene var positivt assosiert med utbrenthet, mens jobbressursene
hadde var positivt assosiert med jobbengasjement. Denne studien viser
også at høyere jobbkrav kan føre til høyere utbrenthet, mens høyere
jobbressurser kan føre til høyere jobbengasjement.

Nevnt i innledningen studerte Jaeggi et al. (2021) effekten av ulikhet i
formue i et småskala samfunn av innfødte i Tsimane i Bolivia hvor det
var 871 observasjoner med i studien, \(n = 871\). I studiet testet de
hvorvidt relativ husholdningrikdom og ulikhet i formue mot forskjellige
psykologiske variabler og helseutfall som depresjon, BMI, blodtrykk og
sykelighet. Dette ble testet mot kontrollvariabler, og studien viste til
en kobling mellom formueulikhet hvor de med lavere formue hadde større
sannsynlighet for å få høyere blodtrykk og luftveissykdommer som kunne
lede til dødsfall. De fant også at de med høyere formue hadde lavere
sannsynlighet for å få depresjon og høyere BMI. Dette kan tyde på at
ulikhet i formue kan ha en negativ effekt på helse og livskvalitet, og
vi vil videre bygge på dette i vår oppgave, for å se om det er en
sammenheng mellom formue og sykefravær i Norge, og om det er andre
faktorer som kan påvirke sykefraværet.

Langseth-Eide \& Vittersø (2021) bygger videre på tidligere forskning og
adresserer limitasjonene ved Job Demands-Resources-modellen. De
argumenterer for at Job Demands-Resources-modellen ved tidligere
forskning har hatt fokus på organisasjonsnivået, og at det er viktig å
se på hvordan Job Demands-Resources-modellen kan brukes bedre på
jobbressurser, jobbengasjement og helserelaterte utfall. De gjorde en
paneldata studie på fast ansatte i Norge med to års tidsforsinkelse med
1533 ansatte første tidsperiode, \(n =1533\) og 1503 ansatte,
\(n = 1503\) neste tidsperiode. Over lengre tid fant de at jobbressurser
hadde en positiv effekt på jobbengasjement, og at jobbengasjement var
negativt assosiert med sykefravær. Dette kan tyde på at høyere
jobbressurser kan føre til høyere jobbengasjement, som igjen kan føre
til lavere sykefravær.

I dagens samfunn er det viktig å forstå hvordan formue og ulikhet kan
påvirke helse, sykefravær og livskvalitet. Dette er spesielt relevant i
lys av den økende formueulikheten som vi har sett de siste årene, ikke
bare i Norge, men i mange vestlige land.

\subsubsection{Makro}\label{makro}

\subsection{Forklar teori og empiriske funn knyttet til koblingen som du
vil undersøke. Være nøye med å gjøre rede for
mekanismer!}\label{forklar-teori-og-empiriske-funn-knyttet-til-koblingen-som-du-vil-undersuxf8ke.-vuxe6re-nuxf8ye-med-uxe5-gjuxf8re-rede-for-mekanismer}

\subsection{Modelloppsett}\label{modelloppsett}

trur kanskje formue effekt er veldig kraftig på ung alder

\subsubsection{Dette er veldig work in
progress}\label{dette-er-veldig-work-in-progress}

Vi tenker vi også å ta å tegne opp en figur som viser hvordan modellen
fungerer i tikz. Oppsettet er veldig work in progress, og mulig vi ender
med 3 ligninger. Men planen er å lage løsninger for forskjellige nivå av
formue, for å så sette dette inn som en faktor i en nyttefunksjon slik
at vi kan tegne indifferensligninger og simplifisere.

\subsubsection{Hovedmodell for sykefravær
(SF)}\label{hovedmodell-for-sykefravuxe6r-sf}

Vi antar at sykefraværet (SF) i hovedsak påvirkes av:

Jobbkrav (JK) (effekten av arbeidsbelastning),

Motivasjon (M) (som en mekanisme/medierende faktor),

Formuenivå (FN) (som hovedprediktor og også direkte påvirker SF),

Så kan vi ha en X som er en mengde kontrollvariabler som for eksempel
avtalte dager, demografi, arbeidsrelaterte forhold osv.

\[
SF_i = \beta_0 + \beta_1 JK_i + \beta_2 M_i + \beta_3 FN_i + \Sigma_j \beta_{4j}X_{ij} + \epsilon_{1i}
\]

Der \(i\) er individet, \(\beta_0\) er konstanten, \(\beta_1\),
\(\beta_2\), \(\beta_3\) er koeffisientene for henholdsvis jobbkrav,
motivasjon og formuenivå, \(\Sigma_j \beta_{4j}X_{ij}\) fanger opp
effekter av eventuelle kontrollvariabler, \(\epsilon_{1i}\) er
feilleddet.

Denne likningen innebærer at formuenivået ikke bare antas å ha en
direkte effekt på sykefravær, men via motivasjon så kan effekten også gå
via en indirekte kanal.

\subsubsection{Ligning for motivasjon
(M)}\label{ligning-for-motivasjon-m}

Motivasjonen antas å bli påvirket av:

Jobbressurser (JR) (dvs. støtte og autonomi i arbeidet),

Formuenivå (FN) (som antas å påvirke hvor sensitiv man er for endringer
i inntekt -- dvs. hvordan man prioriterer fritid/arbeid),

X er kontrollvariabler som f.eks. utdanning eller andre relevante
demografiske/yrkesmessige mål.

\[
M_i = \alpha_0 + \alpha_1 JR_i + \alpha_2 FN_i + \Sigma_k \alpha_{3k}X_{ik} + \epsilon_{2i}
\]

Der \(\alpha_0\) er konstanten, \(\alpha_1\) og \(\alpha_2\) er
koeffisientene for henholdsvis jobbressurser og formuenivå,
\(\Sigma_k \alpha_{3k}X_{ik}\) fanger opp effekter av eventuelle
kontrollvariabler, \(\epsilon_{2i}\) er feilleddet.

putter inn utdanning og alder i z.

Er nokk forskjell på sykefravær på alder ung/gammel. hvor stor forskjell
mellom de på ung og gammel basert på formue

\section{Metode og data}\label{metode-og-data}

I dette kapitlet går vi gjennom datagrunnlag og metode for oppgaven. Vi
vil først forklare hvordan dataene er fremskaffet, så forklare
variablene, og til slutt forklare metoden. Vi vil også gi en innledende
oversikt over dataene, inkludert deskriptiv statistikk for alle
variablene i analysen.

I problemstillingen \emph{forklarer nivået på formue sykefraværet i
Norge?} så velger vi å bruke en Structural Equation Model fordi denne
kan bedre vise oss på hvilken måte formue påvirker sykefraværet og om
det finnes noen indirekte sammenhenger mellom variablene vi velger å
bruke, dette gjør analysen mer kompleks, men vi kan bedre peke direkte
på hvilke effekter som er positive eller negative på selve sykefraværet.

\subsection{Data}\label{data}

Dataen vi bruker er hentet fra Statistisk sentralbyrå (SSB) sin
\href{https://www.ssb.no/arbeid-og-lonn/arbeidsmiljo-sykefravaer-og-arbeidskonflikter/artikler/levekarsundersokelsen-om-arbeidsmiljo-2022}{levekårsundersøkelse
om arbeidsmiljø}, som ble gjennomført i 2022. Vedlagt følger et bilde av
kodeboken:

\includegraphics{dokumentobjekter/bilder/codebook.png}

Statistisk sentralbyrå har gjennomført levekårsundersøkelser siden 1973.
Levekårsundersøkelsen kartlegger arbeidsmiljøforhold blant sysselsatte i
Norge, og tar opp temaer som forhold på arbeidsplassen, fysisk,
ergonomisk og psykososialt arbeidsmiljø, yrkesrelaterte helseplager og
sykefravær og krav og muligheter for selvbestemmelse på jobb.

\subsection{Datakilde og utvalg}\label{datakilde-og-utvalg}

Undersøkelsen er basert på et landsrepresentativt utvalg på 35 345
sysselsatte personer i alderen 18-66 til undersøkelsen i 2022. Utvalget
er tilfeldig trukket fra folkeregisteret, og dataene er samlet inn
gjennom telefonintervjuer og selvadministrert webskjema fra august 2022
til april 2023.

Den totale svarprosenten for undersøkelsen var på 51 prosent, og dataene
er vektet for å være representativt for den norske befolkningen i
alderen 18-66 for å korrigere for noen av skjevhetene i forbindelse med
frafall.

\subsection{Variabler}\label{variabler}

Vi kommer til å bruke flere variabler fra levekårsundersøkelsen for å
analysere sammenhengen mellom formue og sykefravær. Vi vil bruke både
avhengige og uavhengige variabler, latente\footnote{En latent varibel er
  et underliggende, uobserverbart konstrukt som ikke kan måles direkte,
  men som modelleres gjennom flere målbare indikatorer. I SEM tolkes for
  eksempel «motivasjon», «jobbkrav» og «jobbressurser» som latente
  variabler: vi antar at variasjonen i et sett av attestspørsmål
  (indikatorer) reflekterer den samme underliggende faktoren.}
variabler, samt kontrollvariabler for å kontrollere for andre faktorer
som kan påvirke sykefraværet.

\subsubsection{Avhengig og uavhengig
hovedvariabel}\label{avhengig-og-uavhengig-hovedvariabel}

Sykefravær:

Sykefraværssprosent uten feriekorrigering (SF) vil være vår avhengige
variabel, og vi vil bruke sykefraværet i 2022 som mål på sykefravær.
Denne variabelen inneholder sykemeldingsraten selvrapportert av
respondenten i løpet av de siste 12 månedene, og er målt i prosent.

Vi vil også muligens bruke sykefraværsprosent uten feriekorrigering i
2023 som en kontrollvariabel for å se om det er noen endringer i
sykefraværet over tid.

Formue:

Bruttofinanskapital i alt (BF) vil være vår hoveduavhengige variabel, og
vi vil bruke bruttofinanskapital i alt som mål på formue. Denne
variabelen inneholder verdien av alle finansielle eiendeler som
respondenten eier, inkludert kontanter, aksjer, obligasjoner og andre
investeringer og har en maks verdi på 2 500 000.

Vi vil dele denne inn i tre forskjellige tertiler \footnote{Tertiler er
  en statistisk metode for å dele opp et datasett i tre like store
  deler, slik at hver del inneholder en tredjedel av observasjonene.}
for formuegrupper: 0 - 43 333.33, 43 333.33 - 200 000 og 200 000 - 2 500
000. Dette vil gi oss mulighet til å se om det er forskjeller i
sykefravær mellom de forskjellige tertilene. Vi definerer de som lav,
middels og høy formue. Vi vil også bruke log-transformasjon av formue
for å se om det er noen forskjeller i sykefravær mellom de forskjellige
tertilene. Dette kan være nyttig for å se om det er noen ikke-lineære
sammenhenger mellom formue og sykefravær, og for å håndtere høy skjevhet
i dataene, ettersom de fleste har lav formue og få har høy formue.

Vi tror formue spiller inn til hvor sensitiv du er til endringer i
inntekt. Altså ditt konsumnnivå eller etterspurt fritid endrer seg ulikt
basert på om du har mye formue eller ikke. Dette kan være fordi du har
mer buffer til å tåle endringer i inntekt, og dermed kan du være mer
villig til å ta deg fri fra jobb. Og motsatt om du har lite formue så
vil du være mer sensitiv til endringer i inntekt, og dermed vil du være
mer villig til å jobbe mer for å opprettholde inntekten din. Dette kan
føre til at de med høyere formue har lavere sykefravær, mens de med
lavere formue har høyere sykefravær.

\subsubsection{Kontrollvariabler}\label{kontrollvariabler}

Alder:

Alder til respondenten ved utgangen av 2022. Denne kontrollvariabelen
gjør vi ordinal ettersom vi fordeler alderen til respondenten i
aldersgrupper. Vi vil bruke aldersgruppene 18-29, 30-39, 40-49, 50-59 og
60-66 år. Da kan vi påpeke hvis det er forskjeller i sykefravær mellom
de forskjellige aldersgruppene fra unge til eldre personer.

Kjønn:

Kjønn til respondenten. Denne kontrollvariabelen er en dummyvariabel,
hvor 0 er kvinne og referansekategorien 1 er menn. Da vil vi i analysen
direkte se effekten av å være kvinne på sykefraværet.

Utdanning:

Utdanningsnivået til respondenten er en ordinal variabel, og vi vil
bruke utdanningsgruppene grunnskole eller mindre, videregående skole,
Universitet/Høgskole og forskernivå. Vi vil bruke denne variabelen for å
kontrollere for eventuelle utdanningsforskjeller i sykefraværet.

Tilfredshet med arbeid:

Selvrapportert tilfredshet med arbeid (TS) er en ordinal variabel, og vi
vil bruke denne variabelen for å kontrollere for eventuelle forskjeller
i sykefraværet basert på hvor tilfreds respondenten er med jobben sin.
Denne variabelen er målt på en skala fra 1 til 10, hvor 1 er svært
misfornøyd og 10 er svært fornøyd.

Motivasjon:

For variabelen motivasjon bruker vi selvrapportert motivasjon på jobb
(M) som en ordinal variabel, og vi vil bruke denne variabelen for å
kontrollere for eventuelle forskjeller i sykefraværet basert på hvor
motivert respondenten er på jobben sin. Denne variabelen er målt på en
skala fra 1 til 10, hvor 1 er svært lite motivert og 10 er svært
motivert.

Barn:

Antall barn under 18 år i husholdningen som er en kontinuerlig variabel.
Vi vil bruke denne variabelen for å kontrollere for eventuelle
forskjeller i sykefraværet basert på hvor mange barn respondenten har.

Vi vil også mulig bruke dummyvariabler for å kontrollere for andre
faktorer som kan påvirke sykefraværet, som for eksempel yrke, bransje og
arbeidsforhold.

\subsection{Deskriptiv statistikk}\label{deskriptiv-statistikk}

I dette avsnittet vil vi gi en oversikt over deskriptiv statistikk for
alle variablene i analysen. Vi vil presentere gjennomsnitt,
standardavvik og minimums- og maksimumsverdier for alle variablene, samt
korrelasjonsmatrisen for de uavhengige variablene.

I \autoref{tab:deskriptiv} presenteres deskriptiv statistikk for alle
variablene i analysen. Vi ser at sykefraværet i 2022 har et gjennomsnitt
på 12.27 prosent, med et standardavvik på 13.49 prosent. Alder har et
gjennomsnitt på 42.80 år, med et standardavvik på 12.28 år.
Utdanningsnivået har et gjennomsnitt på 4.38, som tilsvarer videregående
skole, med et standardavvik på 1.23.

Av de opprinnelig 17971 inviterte respondentene i datasettet så
fullførte kun 2 080 svarene til alle de relevante variablene. Hvor
eksakt responsrate da blir \(\frac{2080}{17971} = 11.6\%\). Dette kan
føre til skjevheter i dataene, og kan bli en svakhet ved analysen når vi
tolker resultatene. Siden det er vanskelig for oss å vite om det er
systematiske forskjeller mellom de som svarte og de som ikke svarte, så
kan vi ikke si noe sikkert om hvor representativt utvalget er for den
norske befolkningen. Vi blir å sammenlikne alder og kjønn i datasettet
med SSB sine tall for å se om det er noen forskjeller, samt teste
gjennomsnittsalder, og gjennomsnittlig sykefravær for de som svarte og
de som ikke svarte. Hvis det er store forskjeller blir vi å måtte bruke
vektjustering for å korrigere for skjevhetene i dataene.

\begin{table}[ht]
\centering
\begin{tabular}{lrrrrrrr}
\toprule
Variabel & Min & 1.\,Q & Median & Mean & 3.\,Q & Max & N \\
\midrule
Sykefravær 2022                 &   0 &   3 &   7 & 12.27 &  16 &  92 & 2080 \\
Alder                            &  18 &  32 &  43 & 42.80 &  53 &  66 & 2080 \\
Utdanning                        &   2 &   4 &   4 &  4.38 &   6 &   8 & 2080 \\
Kjønn (1=Mann, 2=Kvinne)         &   1 &   1 &   2 &  1.63 &   2 &   2 & 2080 \\
Tilfredshet                      &   1 &   1 &   2 &  2.05 &   3 &   8 & 2080 \\
Motivasjon                       &   1 &   1 &   2 &  2.16 &   3 &   9 & 2080 \\
Barn                             &   0 &   0 &   0 &  0.15 &   0 &   1 & 2080 \\
Støtte fra sjef                  &   1 &   1 &   2 &  2.25 &   3 &   9 & 2080 \\
Støtte fra kollega               &   1 &   1 &   2 &  1.81 &   2 &   9 & 2080 \\
Tilbakemelding fra sjef          &   1 &   2 &   3 &  3.05 &   4 &   9 & 2080 \\
Arbeidsresultater                &   1 &   2 &   2 &  2.59 &   3 &   9 & 2080 \\
Selvbestemmelse (oppgaver)       &   1 &   3 &   3 &  3.25 &   4 &   9 & 2080 \\
Selvbestemmelse (arbeidsinnhold) &   1 &   2 &   2 &  2.48 &   3 &   9 & 2080 \\
Grad arbeidstempo                &   1 &   2 &   3 &  2.87 &   4 &   8 & 2080 \\
Påvirkningsgrad                  &   1 &   2 &   3 &  2.75 &   3 &   9 & 2080 \\
For mye arbeid                   &   1 &   1 &   2 &  1.94 &   2 &   8 & 2080 \\
Høyt arbeidstempo                &   1 &   1 &   2 &  1.78 &   2 &   9 & 2080 \\
Ekstra arbeid                    &   1 &   2 &   4 &  3.43 &   5 &   9 & 2080 \\
\bottomrule
\end{tabular}
\caption{Deskriptiv statistikk for hovedvariabler (N = 2080)}
\label{tab:deskriptiv}
\end{table}

I \autoref{tab:deskr_formue} presenteres deskriptiv statistikk for
sykefravær, alder, motivasjon og tilfredshet etter formuegruppe. Vi ser
at sykefraværet i 2022 har et gjennomsnitt på 12.72 prosent for de med
lav formue, 12.00 prosent for de med middels formue og 12.09 prosent for
de med høy formue. Dette tyder på at det ikke er noen store forskjeller
i sykefraværet mellom de forskjellige formuegruppene. Vi ser også at det
er små forskjeller i alder mellom de forskjellige formuegruppene, der de
med høy formue er eldre enn de med lav og middels formue. Dette kan vise
oss at det er en sammenheng mellom alder og formue, der eldre personer
har høyere formue enn yngre personer.

Motivasjonen er også høyere for de med lav formue enn de med høy formue,
noe som kan si at de med lav formue er mer motivert enn de med høy
formue. Dette kan være fordi de med lav formue har mer å jobbe for, og
derfor er mer motivert til å jobbe hardt. Tilfredsheten er også høyere
for de med lav formue enn de med høy formue, men det er generelt små
forskjeller i tilfredsheten mellom de forskjellige formuegruppene.

\begin{table}[ht]
\centering
\begin{tabular}{lcccccc}
\toprule
 & \multicolumn{2}{c}{Lav formue (n=705)} 
 & \multicolumn{2}{c}{Middels formue (n=704)} 
 & \multicolumn{2}{c}{Høy formue (n=719)} \\
\cmidrule(r){2-3}\cmidrule(lr){4-5}\cmidrule(l){6-7}
Variabel            & M     & SD    & M     & SD    & M     & SD    \\
\midrule
Alder               & 40.72 & 12.28 & 41.67 & 11.81 & 45.94 & 12.07 \\
Motivasjon          &  2.18 &  1.03 &  2.23 &  1.03 &  2.08 &  0.95 \\
Sykefravær 2022     & 12.72 & 13.49 & 12.00 & 14.83 & 12.09 & 14.26 \\
Tilfredshet         &  2.11 &  0.97 &  2.08 &  0.90 &  1.98 &  0.90 \\
\bottomrule
\end{tabular}
\caption{Deskriptiv statistikk etter formuegruppe}
\label{tab:deskr_formue}
\end{table}

I \autoref{tab:deskr_kjonn} presenteres deskriptiv statistikk for
sykefravær etter kjønn. Vi ser at sykefraværet i 2022 har et
gjennomsnitt på 10.92 prosent for menn og 13.06 prosent for kvinner,
kvinner har også høyere sykefravær enn menn. Dette kan skyldes at
kvinner i større grad enn menn jobber i yrker med høyere sykefravær,
eller at kvinner er mer tilbøyelige til å rapportere sykefravær enn
menn. Det kan også være andre faktorer som påvirker sykefraværet, som
for eksempel alder, utdanning og arbeidsforhold. Vi ser også at vi har
en overvekt av kvinner i utvalget, der 63.1 prosent av respondentene er
kvinner og 36.9 prosent er menn. Dette viser oss at det er en skjevhet i
utvalget, der kvinner er overrepresentert i forhold til menn.

\begin{table}[ht]
\centering
\begin{tabular}{lrrrr}
\toprule
Kjønn   & N   & \%   & Gj.snitt sykefravær & SD    \\
\midrule
Mann    & 785 & 36.9 & 10.92               & 13.04 \\
Kvinne  & 1343 & 63.1 & 13.06               & 14.79 \\
\bottomrule
\end{tabular}
\caption{Deskriptiv statistikk for sykefravær etter kjønn (N = 2 128)}
\label{tab:deskr_kjonn}
\end{table}

I \autoref{tab:deskr_utdanning} presenteres deskriptiv statistikk for
sykefravær etter utdanningsnivå. Vi ser at sykefraværet i 2022 har et
gjennomsnitt på 12.82 prosent for de med grunnskole eller mindre, 12.46
prosent for de med videregående skole og 11.67 prosent for de med
universitet/høgskole. Dette tyder på at sykefraværet er høyere for de
med lavere utdanning, og at det kan være er en sammenheng mellom
utdanningsnivå og sykefravær.

\begin{table}[ht]
\centering
\begin{tabular}{lrrrr}
\toprule
Utdanningsnivå                & N   & \%   & Gj.snitt sykefravær & SD    \\
\midrule
Grunnskole eller mindre       & 369 & 17.3 & 12.82               & 15.56 \\
Videregående                   &1074 & 50.5 & 12.46               & 14.26 \\
Universitet/Høgskole           & 685 & 32.2 & 11.67               & 13.31 \\
\bottomrule
\end{tabular}
\caption{Deskriptiv statistikk for sykefravær i 2022 etter utdanningsnivå (N = 2 128).}
\label{tab:deskr_utdanning}
\end{table}

I \autoref{fig:histogram} presenteres histogram og tetthetskurve for
sykefraværet i 2022. Vi ser at sykefraværet er høyreskjevt, med en
høyere andel av respondentene som har lavt sykefravær enn de som har
høyt sykefravær både på menn og kvinner. Vi vet fra
\autoref{tab:deskr_kjonn} at gjennomsnittet for begge kjønn er på
omtrent 11 prosent for menn mens det er på 13 prosent for kvinner, noe
som gjenspeiles i grafen. Det er vanskelig å se, men det er også noen
uteliggere hvor flere respondenter har mer enn 40 prosent sykefravær på
både menn og kvinner.

\begin{figure}[H]
\caption{Histogram og tetthetskurve for sykefravær i 2022}
\label{fig:histogram}
\centering
\includegraphics[width=0.8\textwidth]{dokumentobjekter/figurer/fig_1.png}
\end{figure}

Når vi ser på aldersfordelingen i \autoref{fig:histogram} så ser vi at
den er jevn og symmetrisk fordelt blant respondentene. Som nevnt
tidligere så er spennet på alderene til respondentene i undersøkelsen
mellom 18 til 66 år. Medianalderen kan man se i den blå stiplede linjen
som er på 43 år for menn og 44 år for kvinner.

\begin{figure}[H]
\caption{Histogram og tetthetskurve for alder}
\label{fig:histogram}
\centering
\includegraphics[width=0.8\textwidth]{dokumentobjekter/figurer/fig_2.png}
\end{figure}

For analysen så har vi fordelt alder inn i breddeintervaller på omtrent
10 år, og aldersgruppene er delt inn i 18-29, 30-39, 40-49, 50-59 og
60-66 år. I \autoref{fig:barplot} presenteres et barplot av
aldersgruppene. Vi ser at det er flest respondenter i aldersgruppen
50-59 år med 27.4 prosent, og at det er færrest respondenter i
aldersgruppen 60-66 år med 8.3 prosent. Dette fordi det er aldersgruppen
som er fordelt inn i det laveste breddeintervallet. Ellers er det jevnt
fordelt mellom de andre aldersgruppene, der aldersgruppen 40-49 år har
22.7 prosent, aldersgruppen 30-39 år har 23.5 prosent og aldersgruppen
18-29 år har 17.9 prosent.

\begin{figure}[H]
\caption{Aldersgruppefordeling}
\label{fig:barplot}
\centering
\includegraphics[width=0.8\textwidth]{dokumentobjekter/figurer/fig_3.png}
\end{figure}

I \autoref{fig:histogram_formue} presenteres histogram og tetthetskurve
for bruttofinanskapitalen log-transformert \((1 + x)\) \footnote{Log
  \((1 + x)\) er en vanlig transformasjon for å håndtere høyreskjevhet i
  data, og det kan bidra til å stabilisere variansen og gjøre dataene
  mer normale. Logaritmen gjør at de store verdiene blir mindre og de
  små verdiene blir større.}. Originalt er formuefordelingen høyreskjev,
med en høyere andel av respondentene som har lav formue enn de som har
høy formue noe som kan svekke analysen. Derfor må vi log-transformere
formuefordelingen for å få en mer normalfordelt fordeling både for menn
og kvinner. Når man log-transformerer \((1 + x)\) så tar vi logaritmen
av formueverdiene og legger til 1 for å unngå problemer med nullverdier.

\begin{figure}[H]
\caption{Fordeling av log-transformert bruttofinanskapital}
\label{fig:histogram_formue}
\centering
\includegraphics[width=0.8\textwidth]{dokumentobjekter/figurer/fig_4.png}
\end{figure}

Når vi ser på fordelingen av formue- og utdanningsgrupper fordelt på
kjønn i \autoref{fig:barplot_2} så ser vi at det er flest kvinner i
utdanningsgruppen videregående skole med 46.7 prosent, og det samme
gjelder for menn med 56.9 prosent. Mer kvinner enn menn har
universitetsutdannings eller høyere med 40.2 prosent mot kun 18.5
prosent for menn. Menn har også lavest utdanningsnivå med 24.6 prosent i
utdanningsgruppen grunnskole eller mindre, mens kvinner har 13.1 prosent
i den samme utdanningsgruppen. Dette viser oss at menn har lavere
utdanningsnivå enn kvinner, og at kvinner er mer tilbøyelige til å ta
høyere utdanning enn menn.

Formuefordelingen er delt inn i tertiler, som gjør slik at fordelingen
blir jevnt blant de forskjellige formuegruppene både for menn og
kvinner.

\begin{figure}[H]
\caption{Fordeling av formue- og utdanningsgrupper fordelt på kjønn}
\label{fig:barplot_2}
\centering
\includegraphics[width=0.8\textwidth]{dokumentobjekter/figurer/fig_5.png}
\end{figure}

I \autoref{fig:boxplot} presenteres et boksplott av sykefravær etter
formuegruppe. Vi kan se at det ikke er store forskjeller i sykefraværet
mellom formuegruppene. Medianen vises i den sorte streken i midten av
boksen, og den viser at sykefraværet med små marginer går ned fra lav
formue, til middels formue og til høy formue. Bunnen og toppen til
boksene viser oss henholdsvis første og tredje kvartil, og de stiplede
linjene viser oss minimum og maksimum sykefravær. Det er også noen
uteliggere som er vist med små prikker, og de viser at det er noen
respondenter som har rapportert sykefravær på over 40 prosent. Dette kan
være at de har vært sykemeldt i en lengre periode. I bakgrunnen av
figuren man man se alle observasjonene spredt utover for en bedre
oversikt siden det er mange observasjoner som går over hverandre i
boksen. Dette er gjort med en funksjon som sprer ut observasjonene litt
for å få en bedre oversikt over dem.

\begin{figure}[H]
\caption{Boksplott av sykefravær etter formuegruppe}
\label{fig:boxplot}
\centering
\includegraphics[width=0.8\textwidth]{dokumentobjekter/figurer/fig_6.png}
\end{figure}

I \autoref{fig:boxplot_2} presenteres et boksplott av sykefravær etter
utdanningsnivå. Vi ser at sykefraværet er veldig jevnt mellom
utdanningsnivåene. Medianen er litt over 10 prosent, og som tidligere
vet vi at gjennomsnittlig sykefravær er lavere for høyt utdannede og
litt lavere for de med lavere utdanning.

\begin{figure}[H]
\caption{Boksplott av sykefravær etter utdanningsnivå}
\label{fig:boxplot_2}
\centering
\includegraphics[width=0.8\textwidth]{dokumentobjekter/figurer/fig_7.png}
\end{figure}

Korrelationheatmap om vi får tid her til latente variabler.

\subsection{Metode}\label{metode}

I oppgaven vil vi bruke en kvantitativ metode for å analysere
sammenhengen mellom formue og sykefravær. Vi vil bruke en Structural
Equation Model (SEM) for å teste hypotesene våre, og vi vil kontrollere
for andre relevante faktorer som kan påvirke sykefraværet. SEM er en
statistisk metode som gjør det mulig å teste komplekse modeller med
flere variabler, og som kan håndtere både direkte og indirekte
sammenhenger mellom variablene. Vi vil bruke R for å gjennomføre
analysen, og vi vil bruke pakker som x og x for å implementere
SEM-modellen.

\subsection{Structural Equation Model
(SEM)}\label{structural-equation-model-sem}

Formue inngår i modellen på tre måter: som en direkte
forklaringsvariabel for sykefravær, som en indirekte påvirkning via
motivasjon, og som en modererende variabel som endrer effekten av
jobbkrav og jobbressurser.

Vi antar at formue fungerer som et mål på økonomisk trygghet og
handlingsrom. Personer med høyere formue har trolig mer fleksibilitet
til å håndtere belastninger på jobb, og vil kunne tåle høye jobbkrav
uten samme negative effekt på helse og fravær. Samtidig antar vi at
høyere formue gir høyere jobbmotivasjon fordi økonomisk trygghet gjør
det lettere å finne mening, utvikling og balanse i arbeidet.

På bakgrunn av dette har vi inkludert interaksjonsledd mellom formue og
jobbkrav (\(JD_i FN_i\)), samt mellom formue og jobbressurser
(\(JR_i FN_i\)), for å fange opp slike modererende effekter. Vi har også
modellert motivasjon som en medierende variabel, hvor formue kan påvirke
motivasjonen, som igjen kan påvirke sykefravær.

Dette modellvalget bygger videre på JD-R-rammeverket, men inkluderer
økonomisk kontekst som en faktor som kan endre hvordan individer
påvirkes av jobbsituasjonen. Ved å bruke en SEM-modell kan vi teste både
de direkte og indirekte sammenhengene mellom formue og sykefravær.

\subsubsection{Ligning til modellen}\label{ligning-til-modellen}

\[
SF_i = \beta_0 + \beta_1 JK_i + \beta_2 JR_i + \beta_3 FN_i + \beta_4 (JD_iFN_i) + \beta_5 (JR_i FN_i) + \beta_6 M_i + \Sigma_j \gamma_{j}X_{ij} + \epsilon_{1i}
\]

\[
M_i = \alpha_0 + \alpha_1 JR_i + \alpha_2 FN_i + \Sigma_k \alpha_{3k}X_{ik} + \epsilon_{2i}
\]

\subsubsection{Forklaring av alle deler i
modellen}\label{forklaring-av-alle-deler-i-modellen}

\begin{table}[H]
\centering
\begin{tabular}{lr}
\toprule
Symbol & Forklaring \\ 
\midrule
$SF_i$ & Prosentandel av avtalte arbeidsdager arbeidstaker i er fraværende (sykefravær) \\
$JD_i$ & Latent jobbkrav score (høyere = mer krav) \\
$JR_i$ & Latent jobbressurser score (høyere = mer støtte/autonomi) \\
$FN_i$ & Logaritmen eller prosentil rangeringen av individets (eller husholdningens) formue \\
$M_i$ & Motivasjons-/engasjements score \\
$X_{ij}$ & Kontrollvariabler (alder, kjønn, utdanning …), alle gjennomsnittssentrert \\
$\epsilon_{1i}, \epsilon_{2i}$ & Forstyrrelser (null-gjennomsnitt, ukorrelerte med prediktorer) \\  
$\alpha_{3j} $ & Koeffisienter for kontrollvariablene på Motivasjon i motivasjonsmodellen \\
$\gamma_{j} $ & Koeffisienter for kontrollvariablene i på sykefravær i sykefraværmodellen \\
\hline
\end{tabular}
\caption{Oversikt over variabler i modellen}
\label{tab:variabler}
\end{table}

\subsubsection{Beskrivning av metode}\label{beskrivning-av-metode}

Fra ligningen til vår medierende variabel Motivasjon, forventer vi at
\(\alpha_1 > 0\) og \(\alpha_2 > 0\), som betyr at høyere jobbressurser
og formuenivå vil føre til høyere motivasjon.

\subsubsection{Hypoteser}\label{sec-hypot}

Fra \[
SF_i = \beta_0 + \beta_1 JK_i + \beta_2 JR_i + \beta_3 FN_i + \beta_4 (JD_i FN_i) + \beta_5 (JR_i FN_i) + \beta_6 M_i + \Sigma_j \gamma_{j}X_{ij} + \epsilon_{1i}
\] Forventer vi at: H1: \(\beta_1 > 0\) (høyere jobbkrav gir høyere
sykefravær)

H2: \(\beta_2 < 0\) (høyere jobbressurser gir lavere sykefravær)

H3: \(\beta_3 < 0\) (høyere formuenivå gir lavere sykefravær)

Av de modererende variablene forventer vi at H4: \(\beta_4 < 0\) Høyere
formuenivå demper de negative effektene til høyere jobbkrav. \[
\frac{\partial SF_i}{\partial JD_i} = \beta_1 + \beta_4 FN_i
\]

H5: \(\beta_5 < 0\) Høyere formuenivå forsterker de positive effektene
til høyere jobbressurser. \[
\frac{\partial SF_i}{\partial JR_i} = \beta_2 + \beta_5 FN_i
\]

Vi forventer og at det er en indirekte effekt fra formue igjennom
motivasjon \(\alpha_2 \beta_6\) hvor vi forventer at \(\alpha_2 > 0\) og
\(\beta_6 < 0\) som betyr at høyere formuenivå gir høyere motivasjon,
som igjen gir lavere sykefravær.

\newpage

\section{Analyse}\label{analyse}

\subsubsection{Tabell med resultat fra
regresjonsanalysen(e)}\label{tabell-med-resultat-fra-regresjonsanalysene}

\subsubsection{Redegjørelse for resultat knyttet til
hypoteser}\label{redegjuxf8relse-for-resultat-knyttet-til-hypoteser}

\subsubsection{Redegjørelse for effekt av
kontrollvariabler}\label{redegjuxf8relse-for-effekt-av-kontrollvariabler}

\subsubsection{Redegjørelse for svakheter i
modellen/data}\label{redegjuxf8relse-for-svakheter-i-modellendata}

\newpage

\section{Resultat}\label{resultat}

Her presenteres den empiriske analysen og dens resultater. Vanligvis vil
en empirisk analyse bestå av en regresjonsanalyse med flere variabler.
Andre muligheter kan diskuteres med veilederen.

\subsection{Tabeller}\label{tabeller}

\subsection{Figurer}\label{figurer}

\subsection{Forklaring av tabeller og
figurer}\label{forklaring-av-tabeller-og-figurer}

\newpage

\section{Diskusjon}\label{diskusjon}

Dette kapitlet drøfter resultatene i forhold til problemstillingen. Hva
er funnet ut av, hva gjenstår, hvilke styrker og svakheter har analysen?

\subsubsection{Oppsummering av hva formålet med oppgaven var, og hva
analysen
viste}\label{oppsummering-av-hva-formuxe5let-med-oppgaven-var-og-hva-analysen-viste}

\subsubsection{Diskusjon av hvilke konklusjoner som kan trekkes fra
dette og om resultatene er forenlig med tidligere
funn/teori}\label{diskusjon-av-hvilke-konklusjoner-som-kan-trekkes-fra-dette-og-om-resultatene-er-forenlig-med-tidligere-funnteori}

\subsubsection{Diskusjon av svakheter i
analysen}\label{diskusjon-av-svakheter-i-analysen}

\subsubsection{Diskusjon av implikasjoner for policy gitt
svakheter}\label{diskusjon-av-implikasjoner-for-policy-gitt-svakheter}

\subsubsection{Eventuelt: diskusjon av hva framtidig forskning kan
forske videre på (basert påderes funn og svakheter i
analysen)}\label{eventuelt-diskusjon-av-hva-framtidig-forskning-kan-forske-videre-puxe5-basert-puxe5deres-funn-og-svakheter-i-analysen}

\newpage

\section*{Vedlegg}\label{vedlegg}

Her legger vi til vår QMD fil.

\section*{Appendiks}\label{appendiks}

\subsection*{Kode}\label{kode}

\subsection*{Tester}\label{tester}

\subsection*{Kunstig intelligens}\label{kunstig-intelligens}

\newpage

\section*{Referanser}\label{referanser}

\clearpage

\phantomsection\label{refs}
\begin{CSLReferences}{1}{0}
\bibitem[\citeproctext]{ref-jaeggi2021wealth}
Jaeggi, A. V., Blackwell, A. D., Von Rueden, C., Trumble, B. C.,
Stieglitz, J., Garcia, A. R., Kraft, T. S., Beheim, B. A., Hooper, P.
L., Kaplan, H., et al. (2021). Do wealth and inequality associate with
health in a small-scale subsistence society? \emph{Elife}, \emph{10},
e59437.

\bibitem[\citeproctext]{ref-langseth2021ticket}
Langseth-Eide, B. \& Vittersø, J. (2021). Ticket to ride: A longitudinal
journey to health and work-attendance in the jd-r model.
\emph{International Journal of Environmental Research and Public
Health}, \emph{18}(8), 4327.

\bibitem[\citeproctext]{ref-schaufeli2004job}
Schaufeli, W. B. \& Bakker, A. B. (2004). Job demands, job resources,
and their relationship with burnout and engagement: A multi-sample
study. \emph{Journal of Organizational Behavior: The International
Journal of Industrial, Occupational and Organizational Psychology and
Behavior}, \emph{25}(3), 293--315.

\bibitem[\citeproctext]{ref-ssb2024beregnet}
SSB. (2017). \emph{Beregnet bruttofinanskapital}.
\url{https://www.ssb.no/a/metadata/conceptvariable/vardok/3449/nb}

\bibitem[\citeproctext]{ref-vander2016job}
Vander Elst, T., Cavents, C., Daneels, K., Johannik, K., Baillien, E.,
Van den Broeck, A. \& Godderis, L. (2016). Job demands--resources
predicting burnout and work engagement among belgian home health care
nurses: A cross-sectional study. \emph{Nursing Outlook}, \emph{64}(6),
542--556.

\end{CSLReferences}

\clearpage

\appendix

\section {Appendix Generell KI bruk}

I løpet av koden så kan det ses mange \# kommentarer der det er skrevet
for eks ``\#fillbetween q1 and q2''. Når vi skriver kode i Visual Studio
Code og Rstudio så er det en plugin som heter Github Copilot. Når vi
skriver slike kommentarer eller bare skriver kode så kan den foresøke å
fullføre kodelinjene mens vi skriver de. Noen ganger klarer den det, men
andre ikke. Det er vanskelig å dokumentere hvert bruk der den er brukt
siden det ``går veldig fort'' men siden vi ikke har fått på plass en
slik dokumentasjon så kan all kode der det er brukt kommentarer antas
som at det er brukt Github Copilot. Nærmere info om dette KI verktøyet
kan ses på \url{https://github.com/features/copilot}

\clearpage

\section{notater}\label{notater}

er det avvik mellom fastsatt arbeidstid og hvor mye folk arbeider?

Er folk overarbeidet?

Hvordan har antall legebesøk endret seg samtidig som legemeldt
sykefravær har endret seg. er leger overarbeidet og skriver ut for mye
sykefravær?

Er sykefraværet et problem? Hvordan har sysselsettingsrate endret seg
med sykefravær? er det spesiell korrelasjon mellom egenmeldt og
legemeldt der?

Dårlig ledelse og lite engasjerte arbeidere?

https://www.dagensperspektiv.no/synspunkt/benedicte-langseth-eide-svarer-hr-norge-om-sykefravaer-og-ledelse/1262876

https://www.nord24.no/nar-bedriftene-sliter-med-hoyt-sykefravar-ringer-de-benedicte-disse-tiltakene-nytter/s/5-32-197683

https://www.mdpi.com/1660-4601/18/8/4327

The results provide longitudinal evidence that two well-established job
resources (i.e., social support and feedback) predicted work engagement,
that work engagement was negatively related to sick leave and that this
relation was mediated by subjective health. By showing that
health-related indicators could also be outcomes of the motivational
process in the JD-R model, we have strengthened the model.

https://munin.uit.no/handle/10037/15801

The results also revealed that both workaholics and work-engaged
employees put in more hours at work than was expected of them. We found
that workaholism was negatively related to work-related health, whereas
work engagement was positively related to work-related health. These
findings support the notion of workaholism and work engagement as two
different forms of working hard.

Kanskje en form for ``intensitet'' i hvor sensitiv du er.

Jeg tror formue spiller inn til hvor sensitiv du er til endringer i
inntekt. Altså ditt konsumnnivå eller etterspurt fritid endrer seg ulikt
basert på om du har mye formue eller ikke. Dette kan være fordi du har
mer buffer til å tåle endringer i inntekt.

trur vi blir å få noe bue på den effekten. fattige, vanlige, rike,
megarike vil ha ulik effekt av motivasjon og sånt. e du megarik så har
det jo ingenting og si, e du syk eller vil ta fri så blir du hjemme, men
samtidig så vil du kanskje være spesielt sensitiv om du e fattig og at
om du da e syk eller vil ta fri så vil du både ha dårligere utgangspunkt
i jobbtype og sånt, og også kunne rett å slett være mer syk

mens de i midten rundt ``vanlige'' mot bare rike kan ha 0 effekt, men
kommer vel an på kor mye man ska mene formue har å si til hvor sensitiv
du er til endringer eller potentielle endringer i inntekt derfor æ
tenkte å bare ha det til å være en funksjon av formue kunne være enklere
motivasjon og sånt altså både på bunn og på topp så vil du også ha økt
den stygge m'en ved at du får statlige overføringe som fattig men mye
kapitalfortjeneste som rik

så formue har effekt på hvor mye utdanning du har. formue har effekt på
hvilken motivasjon du har. formue har effekt på m som er annen inntekt
utenom jobb.

g = formue, j = alder, k = utdanning, l = motivasjon

\begin{itemize}
\tightlist
\item
  v = dummyvariabel
\end{itemize}

\[
t^a = h^* - \alpha w - \beta(m(g) + h^*w) - (k\cdot v+j\cdot v)
\]

Dummy variabler for ulike aldersgrupper. beholde en ligning for alle men
da bruke de dummyvariablene. dermed kunne tolke bare en variabel.

forskjellige typer inntekt påvirke forskjellig i m variabelen.

Grunn til cb er at den er enkel og at vi nesten alltid tar log av
dataen. om vi har 0 variabler så blir det bare tull.

\subsection{Notater}\label{notater-1}

\subsubsection{Inntektsfattigdom og
levekårsfattigdom}\label{inntektsfattigdom-og-levekuxe5rsfattigdom}

https://www.ssb.no/sosiale-forhold-og-kriminalitet/artikler-og-publikasjoner/inntektsfattig-eller-levekaarsfattig

Hva så med en mer absolutt tilnærming i form av et forbruksbudsjett
inkludert faktiske bokostnader? Det enkleste målet som ikke tar hensyn
til verken studenter eller formuende, har omtrent like sterk sammenheng
med levekårsfattigdom som den vi finner ved EU60, og dermed noe sterkere
enn ved OECD50. Ved å holde studenter og/eller formuende utenfor
definisjonen med budsjettilnærming, får vi de samme virkningene som vi
har sett tidligere. Det å holde formuende utenfor bidrar til sterkere
sammenheng med levekårsfattigdom, mens det å holde studenter utenfor
ikke gjør det.

Våre funn viser dermed at det ikke er avgjørende om vi definerer
inntektsfattigdom absolutt (ved bruk av husholdningsbudsjett) eller
relativt (ved bruk av ekvivalensskala og inntektsfordeling) når vi ser
på sammenhengen med levekårsfattigdom. Den viktigste faktoren synes å
være at vi tar hensyn til formue, som er en buffer mot mange av
levekårsproblemene. Det har imidlertid ikke særlig betydning å ta hensyn
til studenter i denne sammenhengen, selv om det bidrar til å redusere
andelen inntektsfattige

\subsubsection{Helse og formue}\label{helse-og-formue}

https://pmc.ncbi.nlm.nih.gov/articles/PMC8225390/

controlling for community-average wealth, age, sex, household size,
community size, and distance to markets. Wealthier people largely had
better outcomes while inequality associated with more respiratory
disease, a leading cause of mortality. Greater inequality and lower
wealth were associated with higher blood pressure. Psychosocial factors
did not mediate wealth-health associations. Thus, relative
socio-economic position and inequality may affect health across diverse
societies, though this is likely exacerbated in high-income countries.

\subsubsection{Gatsby curve}\label{gatsby-curve}

``great gatsby curve'' med vedvarende inntekt på tvers av generasjoner.
og siden fattige ikke blir spesielt mye fattigere enn middelklassen, men
at det heller er rikere som flyr fremover. -\textgreater{} kanskje
større forskjell på median og gjennomsnittlig inntekt/formue altså flere
som er ikke rike som jobber i mer sånn lav inntekt yrker og barnehager å
sånt me mye sykdom

\subsubsection{karriærevalg, utdanning
osv.}\label{karriuxe6revalg-utdanning-osv.}

fattigere har dårligere tilgang på ``career role models'' som gjør at de
kanskje ikke vet om de bedre yrkene og sånt og dermed igjen blir mindre
utdanna og sånt
https://www.gallup.com/analytics/506696/amazon-research-hub.aspx

\subsubsection{Stress? glemte studie
her}\label{stress-glemte-studie-her}

inntektsusikkerhet -\textgreater{} økt stress

\subsubsection{Motivasjonseffekt av
ulikhet}\label{motivasjonseffekt-av-ulikhet}

``The motivational cost of inequality: Opportunity gaps reduce the
willingness to work'' https://pmc.ncbi.nlm.nih.gov/articles/PMC7473543/

https://www.brookings.edu/articles/income-inequality-social-mobility-and-the-decision-to-drop-out-of-high-school/

ulikhet gjør at fattige blir mindre motivert siden dem føler det å bli
rik er ``umulig'' og dermed investerer mindre i seg -\textgreater{}
lavere motivasjon og lavere utdanning. kanskje mer fysisk arbeid.

\subsection{Notater}\label{notater-2}

Har høy/lav formue effekt på motivasjonen fra lønnen til arbeid. lav
formue + høy lønn = høy motivasjon? høy formue + høy lønn = ``bryr meg
ikke'' = høy formue+ lav lønn, lav formue + lav lønn = lav motivajon

Kapitalinntekter som rente/aksje osv i forhold til bruttofinanskapital i
alt. kan det være at de med høy formue utenom bolig da har mer andre
inntekter, eller at høy formue bare er lik bolig for mange.

\subsubsection{Formueeffekt på konsum}\label{formueeffekt-puxe5-konsum}

https://fnce.wharton.upenn.edu/wp-content/uploads/2019/08/chodorowreich-crns\_stock\_wealth\_effects.pdf

for hver dollar i formue du har så har du 0.028usd mer i konsum eller
noe

https://usa.visa.com/partner-with-us/visa-consulting-analytics/economic-insights/the-sudden-increase-in-the-wealth-effect-and-its-impact-on-spending.html

så vi kan vise til hvordan de med lav formue da kan være tvungen til å
ta mer tima selv med lav motivasjon for samme konsumnivå fant det
tilfeldigvis her
https://www.economist.com/finance-and-economics/2025/03/19/the-trump-administration-is-playing-a-dangerous-stockmarket-game



\end{document}
